\section{Case Studies} \label{cases_study}

In Section~\ref{sai}, we use a toy language to show that staging an abstract
interpreter is feasible and that a staged artifact can be derived by combining
the staged concrete interpreter and unstaged abstract interpreter.
In this section, we conduct several case studies to further show that this
methodology is practically useful and widely applicable to a diverse set of
analyses.

\subsection{Abstract Compilation \`a la Staging} \label{cs_ac}

We first revisit a similar technique called \textit{abstract compilation} (AC).
AC was introduced by \citet{Boucher:1996:ACN:647473.727587} as an
implementation technique for abstract interpretation-based static analyses.
Similar to the present paper, the idea of AC was inspired by partial evaluation:
the program is known statically and the interpretive overhead can be
eliminated. In AC, the compiled analysis can be represented by either text or
closures (higher-order functions). The minute difference is that the closures
can be executed immediately, while the textual programs need to be compiled and
loaded.

Specifically, \citeauthor{Boucher:1996:ACN:647473.727587} showed how to compile a
monovariant control-flow analysis \cite{Shivers:1991:SSC:115865.115884,
Shivers:1988:CFA:53990.54007} for continuation-passing style (CPS) programs.
The analyzer is a big-step control-environment abstract interpreter. 
By applying AC, every function in the analyzer is refactored to return a
closure only taking an environment argument.  Therefore the overhead of
traversing the AST of input program is eliminated.

We show that AC can be understood and implemented as an instance of staging abstract
interpreters. We first revisit the original implementation of abstract
compilation for 0-CFA, and then reproduce their result by simply adding stage
annotations.  The program generated by our approach provides approximately the
same improvement with regards to speed, but without changing a single line of
the analyzer program relying on type-based staging annotations. With Scala's
type inference, we only need to add staging annotations for function argument
types and return types.
However, AC requires more engineering effort, i.e., rewriting the whole
analyzer into the closure-generation form. Moreover, as shown in
Section~\ref{staged_ds}, our approach is able to not only remove the
interpretive overhead, but also enable several optimizations such as
specialized data structures.

\begin{figure*}
  \centering
  \begin{subfigure}[h]{0.49\textwidth}
    \centering
    \begin{lstlisting}[style=extrasmall]
type CompAnalysis = Store => Store
def compProgram(prog: Expr): CompAnalysis = compCall(prog)
def compCall(call: Expr): CompAnalysis = call match {
  case Letrec(bds, body) =>
    val C1 = compCall(body); val C2 = compArgs(bds.map(_.value))
    ($\sigma$: Store) => C1(C2($\sigma$.update(bds.map(_.name),
       bds.map(b => Set(b.value.asInstanceOf[Lam])))))
  case App(f, args) =>
    val C1 = compApp(f, args); val C2 = compArgs(args)
    ($\sigma$: Store) => C1(C2($\sigma$))
}
def compApp(f: Expr, args: List[Expr]): CompAnalysis = f match {
  case Var(x) => ($\sigma$: Store) =>
    analyzeAbsApp(args, $\sigma$(x), $\sigma$)
  case Op(_) => compArgs(args)
  case Lam(vars, body) =>
    val C = compCall(body)
    ($\sigma$: Store) => C($\sigma$.update(vars, args.map(primEval(_, $\sigma$))))
}
def compArgs(args: List[Expr]): CompAnalysis = args match {
  case Nil => ($\sigma$: Store) => $\sigma$
  case (arg@Lam(vars, body))::rest =>
    val C1 = compCall(body); val C2 = compArgs(rest)
    ($\sigma$: Store) => C2(C1($\sigma$))
  case _::rest => compArgs(rest)
}
  \end{lstlisting}
  \end{subfigure}
\hfill
  \begin{subfigure}[h]{0.49\textwidth}
    \centering
    \begin{lstlisting}[style=extrasmall]
def analyzeProgram(prog: Expr, $\sigma$: Rep[Store]): Rep[Store] =
  analyzeCall(prog, $\sigma$)
def analyzeCall(call: Expr, $\sigma$: Rep[Store]): Rep[Store] =
  call match {
    case Letrec(bds, body) =>
      val $\sigma$_* = $\sigma$.update(bds.map(_.name),
        bds.map(b => Set(b.value.asInstanceOf[Lam])))
      val $\sigma$_** = analyzeArgs(bds.map(_.value), $\sigma$_*)
      analyzeCall(body, $\sigma$_**)
    case App(f, args) => analyzeApp(f, args, analyzeArgs(args, $\sigma$))
  }
def analyzeApp(f: Expr, args: List[Expr], $\sigma$: Rep[Store]):
  Rep[Store] = f match {
    case Var(x) => analyzeAbsApp(args, $\sigma$(x), $\sigma$)
    case Op(_) => analyzeArgs(args, $\sigma$)
    case Lam(vars, body) =>
      val $\sigma$_* = $\sigma$.update(vars, args.map(primEval(_, $\sigma$)))
      analyzeCall(body, $\sigma$_*)
  }
def analyzeArgs(args: List[Expr], $\sigma$: Rep[Store]): Rep[Store] =
  args match {
    case Nil => $\sigma$
    case Lam(vars, body)::rest =>
      analyzeArgs(rest, analyzeCall(body, $\sigma$))
    case _::rest => analyzeArgs(rest, $\sigma$)
  }
  \end{lstlisting}

  \end{subfigure}
  \caption{Comparison of AC (left) and SAI (right). Only core code are shown.}
  \vspace{-1.5em}
  \label{compare_ac_sai}
\end{figure*}

\paragraph{Original AC}
The analysis presented by \citeauthor{Boucher:1996:ACN:647473.727587} is a 0-CFA
for a small CPS language consisting of abstractions, applications, recursion, and
primitive operators. Figure \ref{compare_ac_sai} (left) shows the AC
implementation: different syntactic constructs are handled by different functions
(e.g., @compCall@ and @compApp@).
All original functions take a store as regular argument. After refactoring to
AC, every function returns a value of type @CompAnalysis@, which is a function
value that takes and returns stores. After the first run of the
analysis, we have traversed the AST and obtained a single closure that takes a
store and returns a store. Therefore, the ASTs are treated as static terms, and
the stores are dynamic terms. Finally, we may invoke the compiled closure with an
initial store to complete the analysis.

\paragraph{AC through Staging}

On the other side, our approach works equivalently through staging: the ASTs
are static, whereas stores are dynamic. Therefore, we annotate the type @Store@
with @Rep@ (Figure \ref{compare_ac_sai} (right)). In fact, the only changes of
our approach using the LMS framework are the types of @Store@ to @Rep[Store]@,
since they will be known at the next stage. There is no need for other changes.
The generated programs consist entirely of looks-up and updating operations
w.r.t the store.  The functions at the current stage are eliminated in the
residual program; therefore, we do not generate higher-order functions, which
provides additional performance improvement. Nevertheless, the LMS framework
provides support for necessary data structures, such as for @Map@.  We consider
these efforts reusable and completely orthogonal to the implementation of the
analysis.

%%%%%%%%%%%%%%%%%%%%%%%%%%%%%%%%%%%%%%%%%%%%%%%%%%%%%%%%%%%%%%%%%%%%%%%%%%%%%
%%%%%%%%%%%%%%%%%%%%%%%%%%%%%%%%%%%%%%%%%%%%%%%%%%%%%%%%%%%%%%%%%%%%%%%%%%%%%
\subsection{Control-Flow Analysis} \label{cfa}

To demonstrate that the SAI approach is applicable to various analyses, we now
extend and refactor the staged abstract interpreter we presented in Section
\ref{sai} to several variants of control-flow analyses (CFA). We first discuss
the calling-context sensitive analysis for staged analyzers by tweaking the
allocation strategy. Then, we describe how to implement store-widening with the
staged monads, which is a common optimization in CFA. Finally, we integrate
abstract garbage collection with staging. All of these extensions are orthogonal to
staging.

\subsubsection{Context-Sensitivity}

The abstract addresses we introduced in Section \ref{sai} are identical
to the variable names, which is a context-insensitive schema.  Through tweaking
the address allocation strategy, we can achieve various context-sensitive
analyses \cite{DBLP:conf/icfp/Gilray0M16}. Here, we demonstrate a
calling-context sensitive analysis, i.e., $k$-CFA-like analysis
\cite{DBLP:journals/jfp/HornM12}, can be implemented in our staged abstract
interpreter.
\begin{lstlisting}
  type Time = List[Expr]
  case class KCFAAddr(x: Ident, time: Time) extends Addr
\end{lstlisting}

The timestamp is a finite list of expressions that tracks the $k$-most
recent calling contexts. The definition of abstract addresses @KCFAAddr@ is also
changed to include an identifier and the time when it gets allocated, meaning that
this address points to some values under such calling context. If $k$ is 0, we
obtain a monovariant analysis as demonstrated before; if $k > 0$, we obtain a
family of analyses with increasing precision. Note that when $k = 0$, the
address space is statically determined by the set of variables appeared in the
program, and we may exploit it to optimize the generated code. But when $k > 0$,
the address generation happens at the next stage, i.e., analysis time.
\begin{lstlisting}
  type AnsM[T] = RepReaderT[RepStateT[RepStateT[ // add Time with a RepStateT
                  RepSetReaderStateM[Cache, Cache, ?], Time, ?], Store, ?], Env, T]
  type Result = (Rep[Set[(Value, Store, Time)]], Rep[Cache])
\end{lstlisting}

We then integrate the timestamp into our monad stack by adding a staged
@StateT@ monad. The result type is accordingly updated to contain the set of
tuples, where the field of @Time@ is added.
\begin{lstlisting}
  def tick(e: Expr): AnsM[Unit] = for {
    τ <- get_time; u <- liftM(StateTMonad.put((e::τ).take($k$)))
  } yield u
  def alloc(x: String): AnsM[Addr] = for {
    τ <- get_time
  } yield unit(KCFAAddr(x, τ)) // we use unit to turn a current-stage constant to a next-stage value
\end{lstlisting}

Every time when we call the @eval@ function, we also refresh the timestamp by
calling the @tick@ function, which updates the timestamp in the state monad.
The @tick@ function is implemented as prepending the current control expression
@e@ being evaluated to the existing calling context, and then taking the first
$k$ elements from the list. The type @Config@ used in the caching algorithm is
also changed to include timestamps.

\subsubsection{Store-Widened Analysis}

A commonly-used technique to improve the running time of control-flow analyses
is store-widening. As shown by \citet{Darais:2015:GTM:2814270.2814308,
DBLP:journals/pacmpl/DaraisLNH17}, a store-widening analysis can be easily
implemented in the monadic abstract interpreter by swapping the @StateT@ and
the @SetT@ monad transformers. This optimization is orthogonal to staging.
Within the monadic interpreter framework, we just need to swap the
same \textit{staged} version of monad transformers. After this adjustment, conceptually, we
have the following monad stack:
\begin{lstlisting}
  type AnsM[T] = RepReaderT[RepSetT[RepStateT[
                   RepReaderT[RepStateT[IdM, Cache, ?], Cache, ?], Store, ?], ?], Env, T]
\end{lstlisting}

\subsubsection{Abstract Garbage Collection}

Abstract garbage collection is a technique to reclaim unreachable addresses in
the store while performing the abstract interpretation
\cite{Might:2006:IFA:1159803.1159807}. \citet{DBLP:journals/pacmpl/DaraisLNH17}
showed that, for big-step monadic abstract interpreters, we may add a @Reader@
monad to track the set of root addresses and compute the reachable addresses every
time when we obtain a value. Here, we adopt the idea with staged monads:
\begin{lstlisting}
  type AnsM[T] = RepReaderT[RepReaderT[
                   RepStateT[RepSetReaderStateM[Cache, Cache, ?], Store, ?],
                 Set[Addr], ?], Env, T] //add a reader monad for Set[Addr]
\end{lstlisting}

The skeleton of a staged version of abstract garbage collection implementation
is shown in Figure \ref{fixgc}.
We refactor the @fix@ function with caching (used in Section \ref{sai}) to @fix_gc@
(the caching part is omitted in the presented code snippet). Several auxiliary
functions are used: @ask_root@ retrieves the current root addresses from the
monad stack; @root@ computes the reachable addresses given a value; and @gc@
performs garbage collection according to the given reachable addresses and
returns a new store. We also need to compute the root addresses for compound
expressions, which is elided in the code snippet; for the details of this part,
readers may refer to \citet{DBLP:journals/pacmpl/DaraisLNH17}.

\begin{figure}[h!]
\vspace{-1em}
\begin{lstlisting}
  def fix_gc(ev: (Expr => Ans) => (Expr => Ans))(e: Expr): Ans = for {
    ψ <- ask_roots ... // omit code for retrieving ρ, σ, in and out
    val res: Rep[(Set[(Value,Store)], Cache)] = ...  // omit code for checking caches
    _ <- put_out_cache(res._2);  vs <- lift_nd(res._1)
    σ <- gc(ψ ⊔ root(vs._2));     _ <- put_store(σ)  // gc performs abstract garbage collection
  } yield v
\end{lstlisting}
\vspace{-0.5em}
\caption{\texttt{fix\_gc} for abstract garbage collection} \label{fixgc}
\vspace{-2em}
\end{figure}

%def root(v: Rep[Value]): Set[Addr]  def ask_root: AnsM[Set[Addr]]  def gc(ra: Rep[Set[Addr]]): AnsM[Store]

%%%%%%%%%%%%%%%%%%%%%%%%%%%%%%%%%%%%%%%%%%%%%%%%%%%%%%%%%%%%%%%%%%%%%%%%%%%%%

\subsection{Modular Staged Analysis for Free} \label{modular}

One of the challenges of modern static analysis is that programs usually depend
on large libraries, and analyzing these libraries is time-consuming
\cite{toman_et_al:LIPIcs:2017:7121}.
If we are able to analyze programs and libraries separately and reuse the
results without losing precision, we can reduce the overhead of repeatedly
analyzing libraries. Some static analyzers compute a summary for a function or
a module, which can be reused later. However, when lacking context
information, those analyses can be too conservative or unsound. 
In this section, we show that meta-programming can provide a
trade-off between summary-based modular analyses and slow whole-program
analyses, by utilizing an existing whole-program analyzer.

As we have already seen in Section \ref{sai}, the primitive operation @close@
can denote a $\lambda$-term to a next-stage Scala function. 
The specialization of the staged abstract interpreter to a
$\lambda$-term is also independent of other parts of the analyzed program.
\begin{lstlisting}
  def close(ev: Eval => Ans)(λ: Expr, ρ: Rep[Env]): Rep[Value]
\end{lstlisting}

To illustrate the idea concretely, we consider a library that contains a
function @map@ and use the Scheme language as example:
\begin{lstlisting}
  (define (map xs f) (if (null? xs) '() (cons (f (car xs)) (map (cdr xs) f))))
\end{lstlisting}

The specialization (0-CFA and without store-widening) of @close@ to
@map@ generates the following next-stage Scala code, that is, the
function @x39@:
\begin{lstlisting}
  val x39 = {(x40: List[Set[AbsValue]],          // list of argument values
              x41: Map[Addr, Set[AbsValue]],     // store
              x42: Map[Config, Set[ValSt]],      // in cache
              x43: Map[Config, Set[ValSt]]) =>   // out cache
    val x50 = List[Addr](ZCFAAddr("xs"), ZCFAAddr("f")) // addresses of arguments
    val x51 = x50.zip(x40)                              // addresses paired with their values
    val x61 = x51.foldLeft (x41) { case (x52, x53) =>   // preparing a new store that joins
      val x54 = x53._1;  val x55 = x53._2               //   the arguments and their values
        val x58 = x52.getOrElse(x54, x57)               //   into the latest store, initially x41
        val x59 = x58.union(x55);  x52 + (x54 -> x59) }
    ... }
\end{lstlisting}

We borrow the notation from traditional modular analysis
\cite{DBLP:conf/cc/CousotC02}: such a generated function is a modularly
reusable artifact, i.e., independent of its context.
The generated functions already eliminate the interpretation overhead and
monadic overhead via staging, and can therefore be considered as partial
summaries. However, we still need to run the generated function at the next
stage to obtain the analysis result.  At the next stage, we can provide
different abstract arguments, stores, and caches, depending on the context
where it is called.

Compared with traditional summary-based analysis, our partial
summaries are represented by executable functions that are
parameterized to the context information. In our approach, the
transition from a whole-program analyzer to a modular-staged analyzer is
mechanized and for-free via staging. We do not need to design any
intermediate representation or data structures of summaries.  In some
scenarios, traditional summary-based analyses are able to carry more
information within a module, e.g., the summary of their internal
submodules. The staged approach presented here may still need to run
through the fixed-point for submodules.

\iffalse
\paragraph{Partially-Static Context} Under a polyvariant analysis (like
$k$-CFA), such summary generated by specializing static analyzer even could do
more for calling context by partially-static data optimization. For example,
consider that we have a library consists of three functions @f@, @g@ and @h@,
where @f@ calls @g@ and @h@ internally and returns @y@ finally (we extend the
language to have multiple arguments). @g@ and @h@ do not depend on @f@. Now we
would like to modularly specialize these functions, starting from @f@ with an
initial environment containing @f@, @g@ and @h@.
\begin{lstlisting}
  def f(a, b, c) = val x = g(a, b); val y = h(x, c);  y
\end{lstlisting}

Since $k$ is bounded to a fixed number, after the analysis running $k$ steps
inside @f@, the calling context will purely depend on the static structure of
our module being analyzed.
\fi

%%%%%%%%%%%%%%%%%%%%%%%%%%%%%%%%%%%%%%%%%%%%%%%%%%%%%%%%%%%%%%

\iffalse
Another perspective: programs are data for an abstract interpreter, so if we
have $n$ programs, then maybe there can be $n$ stages. Probably we can analyze
first $m$ programs, and generate a residual abstract interpreter waiting for the
rest $(n-m)$ programs. These $(n-m)$ programs might be (abstract) arguments for
the first $n$ programs, and the abstract interpreter itself might be a partial
abstract interpreter.

\subsection{Numerical Analysis in Imperative Languages} \label{cases_imp}

Now we consider in a first-order imperative language, we may care more about the
data-flow because the control-flow is relatively easy to obtain. In this
section, we show the staging of other abstract domains, particularly an interval
domain for numbers. It has been shown that specializing abstract domains with
respect to the structure of analyzed program significantly improves the
performance: a recent example is online decomposition of polyhedron
\cite{DBLP:conf/popl/SinghPV17, Singh:2017:PCD:3177123.3158143}. In this
section, we first show how to support imperative language features in the
generic abstract interpreter. Then we present a similar idea for the interval
domain and show that the specialization is feasible by staging systematically.

\subsubsection{Scaling to Imperative Languages}

To evaluates an assignment, we first evaluate its right-hand side, and then put
the value into the slot where the address of @v@ points to. For simplicity, we
elect to make the value of the assignment be an @void()@ value.
\begin{lstlisting}
case Assign(x, e) =>
  val (v, $\sigma$_*) = ev(e, $\rho$, $\sigma$)
  (void(), put($\sigma$_*, get($\rho$, x), v))
\end{lstlisting}

To evaluates a @while@ loop statement, we evaluate the condition first. Then
similar to how we treat @branch0@, we have a generic @branch@ function but works
on Boolean values. For the @true@ branch, we recursively call @ev@ on a newly
constructed expression @Seq(e, While(t, e))@ meaning that first evaluates @e@
and the repeat the loop. Otherwise, for the other branch, we simply return a
void value and current store.
\begin{lstlisting}
case While(t, e) =>
  val (tv, t$\sigma$) = ev(t, $\rho$, $\sigma$)
  branch(tv, ev(Seq(e, While(t, e)), $\rho$, t$\sigma$), (void(), t$\sigma$))
\end{lstlisting}

\subsubsection{Staged Interval}

Interval domain is a relative simple domain which consists of two numeric
fields, an upper bound and a lower bound. Here we annotate the upper bound and
lower bound fields to be of type @Rep[Double]@. The operations such as @+@ of
two intervals produce a new
\begin{lstlisting}
case class Interval(lb: Rep[Double], ub: Rep[Double]) {
  def +(that: Interval): Interval = that match {
    case Interval(lb_, ub_) => Interval(lb + lb_, ub + ub_)
  }
  ...
}
\end{lstlisting}

\fi
