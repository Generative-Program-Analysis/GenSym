\section{Case Studies} \label{cases_study}

In Section~\ref{sai}, we have shown that staging an abstract interpreter is
feasible and can be systematic after correctly identifying the binding-times,
despite the fact that the abstract interpreter is intended to be imprecise and
easy to implement. In this section, we conduct several case studies to show that
this approach is also useful and widely applicable to different analyses.

\subsection{Abstract Compilation a la Staging} \label{cs_ac}

\citeauthor{Boucher:1996:ACN:647473.727587} introduced abstract compilation (AC)
as a new implementation technique for abstract interpretation based static
analysis \cite{Boucher:1996:ACN:647473.727587}. The idea is inspired by partial
evaluation similar to the present paper -- the program can be known statically,
the overhead of interpretation can be eliminated. In AC, the compiled analysis
can be represented by either text or closures (higher-order functions); though
the closures can be executed immediately, the textual programs need to be
compiled and loaded first.

Specifically, \citeauthor{Boucher:1996:ACN:647473.727587} show how to compile a
monovariant control-flow analysis \cite{Shivers:1991:SSC:115865.115884,
Shivers:1988:CFA:53990.54007} for continuation-passing style (CPS) programs.
Since the analyzed program is written in CPS, the analyzer is essentially a
big-step control-environment abstract interpreter. Closure generation compiles
the analysis as a closure taking an environment argument. The overhead of
traversing the abstract syntax tree of input program also has been eliminated.

In this section, we show that \citeauthor{Boucher:1996:ACN:647473.727587}'s
abstract compilation can be understood and implemented as an instance of staging
abstract interpreters. We first revisit the original implementation of abstract
compilation of 0-CFA, and then reproduce their result by simply adding stage
annotations. The generated program of our approach improves approximately the
same extent of speed, but without changing a single line of the analyzer program
(with the use of LMS). However, closure generation requires more engineering
effort, specifically a whole-program conversion on the analyzer. Moreover, as
shown in Section~\ref{staged_ds}, our approach is able to not only remove the
interpretive overhead, but also specialize the data structures used in the
analysis, for example, the environment that maps variables to sets of lambda.

\begin{figure*}
  \centering
  \begin{subfigure}[h]{0.49\textwidth}
    \centering
    \begin{lstlisting}[style=extrasmall]
type CompAnalysis = Store => Store
def compProgram(prog: Expr): CompAnalysis = compCall(prog)
def compCall(call: Expr): CompAnalysis = call match {
  case Letrec(bds, body) =>
    val C1 = compCall(body); val C2 = compArgs(bds.map(_.value))
    ($\sigma$: Store) => C1(C2($\sigma$.update(bds.map(_.name), 
       bds.map(b => Set(b.value.asInstanceOf[Lam])))))
  case App(f, args) =>
    val C1 = compApp(f, args); val C2 = compArgs(args)
    ($\sigma$: Store) => C1(C2($\sigma$))
}
def compApp(f: Expr, args: List[Expr]): CompAnalysis = 
  f match {
    case Var(x) => ($\sigma$: Store) => 
      analysisAbsApp($\sigma$.lookup(x), args, $\sigma$)
    case Op(_) => compArgs(args)
    case Lam(vars, body) =>
      val C = compCall(body)
      ($\sigma$: Store) => C($\sigma$.update(vars, args.map(primEval(_, $\sigma$))))
  }
def compArgs(args: List[Expr]): CompAnalysis = args match {
  case Nil => ($\sigma$: Store) => $\sigma$
  case (arg@Lam(vars, body))::rest =>
    val C1 = compCall(body); val C2 = compArgs(rest)
    ($\sigma$: Store) => C2(C1($\sigma$))
  case _::rest => compArgs(rest)
}
  \end{lstlisting}
  \end{subfigure}
\hfill
  \begin{subfigure}[h]{0.49\textwidth}
    \centering
    \begin{lstlisting}[style=extrasmall]
def analyzeProgram(prog: Expr, $\sigma$: Rep[Store]): Rep[Store] = 
  analyzeCall(prog, $\sigma$)
def analyzeCall(call: Expr, $\sigma$: Rep[Store]): Rep[Store] = 
  call match {
    case App(f, args) => analyzeApp(f, args, analyzeArgs(args, $\sigma$))
    case Letrec(bds, body) =>
      val $\sigma$_* = $\sigma$.update(bds.map(_.name), 
        bds.map(b => Set(b.value.asInstanceOf[Lam])))
      val $\sigma$_** = analyzeArgs(bds.map(_.value), $\sigma$_*)
      analyzeCall(body, $\sigma$_**)
  }
def analyzeApp(f: Expr, args: List[Expr], $\sigma$: Rep[Store]): 
  Rep[Store] = f match {
    case Var(x) => analyzeAbsApp(args, $\sigma$(x), $\sigma$)
    case Op(_) => analyzeArgs(args, $\sigma$)
    case Lam(vars, body) =>
      val $\sigma$_* = $\sigma$.update(vars, args.map(primEval(_, $\sigma$)))
      analyzeCall(body, $\sigma$_*)
  }
def analyzeArgs(args: List[Expr], $\sigma$: Rep[Store]): Rep[Store] = 
  args match {
    case Nil => $\sigma$
    case Lam(vars, body)::rest => 
      analyzeArgs(rest, analyzeCall(body, $\sigma$))
    case _::rest => analyzeArgs(rest, $\sigma$)
  }
  \end{lstlisting}

  \end{subfigure}
  \caption{Comparison of AC (left) and SAI (right). Only core code are shown.}
  \label{compare_ac_sai}
\end{figure*}

\subsubsection{Closure Generation}

The analysis presented by \citeauthor{Boucher:1996:ACN:647473.727587} is 0-CFA
for a CPS language consisting of lambda terms, applications, @letrec@ and
primtive operators. The analyses for different syntactic constructs are
decomposed into different functions, such as @analyzeCall@ and @analyzeApp@. The
idea of closure generation is to rewrite these functions. Where previously they
may take both static arguments and dynamic arguments, after the rewrite only the
static arguments is taken. In this case, the static arguments are syntactic
terms; the dynamic arguments are stores. After written in AC style, functions
like @compCall@ (compiled version of @analyzeCall@) return a value of type
@CompAnalysis@, i.e., a closure that takes a store and returns a store. The
result of multiple calls on such functions, for example, @compCall@ and
@compArgs@, can be composed. The generated closure only takes stores, because
the input program is specialized into the closure. The code is shown in
Figure~\ref{compare_ac_sai} (left).

\subsubsection{Staged 0-CFA}

On the other side, our approach does exactly the same thing through staging: the
syntactic terms are static, and stores are dynamic, therefore the generated code
just looks-up and updates the store. Figure~\ref{compare_ac_sai} (right) shows
the code written with LMS. In fact, the only changes are the type of @Store@ is
replaced with @Rep[Store]@ indicating that the values of type @Store@ will be
known at the next stage. Indeed, additional engineering efforts are required to
make this happen, including: the staged version @Map@ which is already included
in LMS; implicit @lift@ function that transform a current-stage constant value
to next stage; next-stage representation of the syntactic terms, i.e., proper
@toString@ functions of AST structs. We consider these efforts relatively small,
and they do not interfere the actual analysis we desire.

%%%%%%%%%%%%%%%%%%%%%%%%%%%%%%%%%%%%%%%%%%%%%%%%%%%%%%%%%%%%%%%%%%%%%%%%%%%%%
%%%%%%%%%%%%%%%%%%%%%%%%%%%%%%%%%%%%%%%%%%%%%%%%%%%%%%%%%%%%%%%%%%%%%%%%%%%%%

\subsection{Control-flow Analysis} \label{cfa}

The target language we presented in Section~\ref{bg_lang} is essentially a
higher-order functional language. One fundamental analysis task for functional
programs is control-flow analysis, i,e., determining which functions will
possibly be applied at each call-site. The abstract interpreter we used in
Section~\ref{unstaged_abs} and Section~\ref{sai} is a store-widened 0-CFA-like
abstraction, moreover it is also a pushdown control-flow analysis; in the last
section, we also reviewed AC with finite-state 0-CFA. In this section, based on
the existing staged abstract interpreter, we further develop the staging
techniques with control-flow analyses, including recoverying a more precise
store model and adding context-sensitivity to the analysis.

\subsubsection{A More Precise Store Model}

The store-widened analysis we implemented in Section~\ref{sai} uses a single
store to approximate the runtime store. It can be efficently computed in
polynomial time together with 0-CFA-like abstraction, but sometimes we may
desire a more precise result that distinguishes the final values and stores for
different closure targets. To achieve this goal, we need to tweak our abstract
interpreter and type instantiation. The answer type @Ans@ is changed to a set of
@VS@s where a @VS@ is a pair of sets of abstract values (such as closures or
abstract numbers) and a store.

\begin{lstlisting}
type VS = (Set[AbsValue], Store)
type Ans = R[Set[VS]]
\end{lstlisting}

Note that the type @Ans@ uses our stage polymorphic type @R@, meaning that under
staging the type @Ans@ represents a next stage value. Then, our generic
interpreter is also changed when handling function applications.

\begin{lstlisting}
case App(e1, e2) =>
  val e1ans = ev(e1, $\rho$, $\sigma$)
  val e1vs  = choices(e1ans)
  val e2ans = ev(e2, $\rho$, e1vs.$\sigma$)
  val e2vs  = choices(e2ans)
  apply_closure(ev)(e1vs.v, e2vs.v, e2vs.$\sigma$)
\end{lstlisting}

The idea to handle the application is to explore all possible closures from @e1@
and meanwhile use the latest store. What @choices@ does is similar to McCarthy's
@amb@ operator ~\cite{MCCARTHY196333}: it non-deterministically returns an
element of type @VS@ from its argument, e.g., @e1ans@, to its right-hand side
receiver, i.e., @e1vs@. The function @choices@ internally uses the delimited
control operator @shift@ to capture the continuation, which is the code block
after its call site. This allows us to perform nesting depth-first evaluation
for function application while still writing the program in direct-style
\cite{Wei:2018:RAA:3243631.3236800}. Again, we can stage this part as we did for
the naive abstract interpreter.

\subsubsection{Context-Sensitivity}

We add k-CFA-like context-sensitivity to the analysis by introducing an abstract
timestamp, whose concrete instantiation is a finite list of expressions that
track $k$ recent calling contexts. The definition of abstract addresses is
changed to a tuple of identifiers and the time it get allocated, meaning that
this address points to some values under such calling context. If $k$ is 0, we
obtain a monovariant analysis as demonstrated before; if $k > 0$, we obtain a
family of analysis with increasing precision.

\begin{lstlisting}
type Time = List[Expr]
\end{lstlisting}

Every time when we call the @eval@ function, we refresh the timestamp by calling
function @tick@, which returns a new timestamp. Here we adopt a $k$-CFA-like
allocation strategy, therefore the @tick@ function can be implemented as
appending the current expression being evaluated to the existing calling
context, and then taking the first $k$ elements from the list.

\begin{lstlisting}
def tick(e: Expr, $\tau$: R[Time]): R[Time] = (e :: $\tau$).takewhile
def eval(ev: EvalFun)
  (e: Expr, $\rho$: R[Env], $\sigma$: R[Store], $\tau$: R[Time]): Ans = {
    val $\tau$_* = tick(e, $\tau$)
    e match {
      case Lit(i) => ...
      ...
    }
  }
\end{lstlisting}

Accordingly, the type of return value is accompanied by the timestamp. For a
recursive call of @ev@, which would also return the latest timestamp, and that
timestamp will be used for the evaluation afterward.

\begin{lstlisting}
type VST = (Value, Store, Time)
type Ans = R[Set[VST]]
\end{lstlisting}

Using other allocation strategies to achieve different sensitivities is also
possible \cite{DBLP:conf/icfp/Gilray0M16} and can be staged under our framework.

%%%%%%%%%%%%%%%%%%%%%%%%%%%%%%%%%%%%%%%%%%%%%%%%%%%%%%%%%%%%%%%%%%%%%%%%%%%%%

\subsection{Modular Analysis for Free}

% \subsubsection{Mixed Sensitivity}
% using different k?

One of the challenges of modern static analysis is program usually depends on
large libraries programs~\cite{toman_et_al:LIPIcs:2017:7121}. Can we analyze
programs and libraries separately and reuse the result without losing precision?
Then we can reduce part of the overhead of repeatedly analyzing libraries for
different programs. Indeed, some static analyzers compute summary for a function
or a module, which can be reused later, however they are mostly too conservative
or unsound, which both lead to imprecision.

The specialization of abstract interpreter provides a chance to obtain such
partial analysis result in a mechanized way, but still keeps the analysis sound.
As we see when compiling the closures, we can specialize the abstract
interpreter with respect the body expression of the lambda term without knowing
the actual argument. The programs with some unknown variables are open programs,
which is exactly the case if we want to analyze programs in a modular way.

For a concrete analysis, for example, $k$-CFA ($k$ > 0) is naturally a
whole-program analysis, because it is inter-procedural and needs the last $k$
calling contexts to distinguish different call sites, where the calling contexts
are dynamic values. However, it is possible to analyze programs (libraries)
separately through specializing an abstract interpreter that generates the
analysis as the next-stage program and leave the unavailable programs and
calling contexts as dynamic parameters, and then install these contexts when we
have the whole program.

\iffalse
Another perspective: programs are data for an abstract interpreter, so if we
have $n$ programs, then maybe there can be $n$ stages. Probably we can analyze
first $m$ programs, and generate a residual abstract interpreter waiting for the
rest $(n-m)$ programs. These $(n-m)$ programs might be (abstract) arguments for
the first $n$ programs, and the abstract interpreter itself might be a partial
abstract interpreter.
\fi

\subsection{Numerical Analysis in Imperative Languages} \label{cases_imp}

Now we consider in a first-order imperative language, we may care more about the
data-flow because the control-flow is relatively easy to obtain. In this
section, we show the staging of other abstract domains, particularly an interval
domain for numbers. It has been shown that specializing abstract domains with
respect to the structure of analyzed program significantly improves the
performance: a recent example is online decomposition of polyhedra
\cite{DBLP:conf/popl/SinghPV17, Singh:2017:PCD:3177123.3158143}. In this
section, we first show how to support imperative language features in the
generic abstract interpreter. Then we present a similar idea for the interval
domain and show that the specialization is feasible by staging systematically.

\subsubsection{Scaling to Imperative Languages}

To evaluates an assignment, we first evaluate its right-hand side, and then put
the value into the slot where the address of @v@ points to. For simplicity, we
elect to make the value of the assignment be an @void()@ value.

\begin{lstlisting}
case Assign(x, e) =>
  val (v, $\sigma$_*) = ev(e, $\rho$, $\sigma$)
  (void(), put($\sigma$_*, get($\rho$, x), v))
\end{lstlisting}

To evaluates a @while@ loop statement, we evaluate the condition first. Then
similar to how we treat @branch0@, we have a generic @branch@ function but works
on boolean values. For the @true@ branch, we recursively call @ev@ on a newly
constructed expression @Seq(e, While(t, e))@ meaning that first evaluates @e@
and the repeat the loop. Otherwise, for the other branch, we simply return a
void value and current store.

\begin{lstlisting}
case While(t, e) =>
  val (tv, t$\sigma$) = ev(t, $\rho$, $\sigma$)
  branch(tv, ev(Seq(e, While(t, e)), $\rho$, t$\sigma$), (void(), t$\sigma$))
\end{lstlisting}

\subsubsection{Staged Interval}

Interval domain is a relative simple domain which consists of two numeric
fields, an upper bound and a lower bound. Here we annotate the upper bound and
lower bound fields to be of type @Rep[Double]@. The operations such as @+@ of
two intervals produce a new

\begin{lstlisting}
case class Interval(lb: Rep[Double], ub: Rep[Double]) {
  def +(that: Interval): Interval = that match {
    case Interval(lb_, ub_) => Interval(lb + lb_, ub + ub_)
  }
  ...
}
\end{lstlisting}

