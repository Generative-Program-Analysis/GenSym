\section{From Abstract Interpreters to Staged Abstract Interpreters} \label{sai}

In the previous sections, we have seen an unstaged abstract interpreter and a
staged concrete interpreter, now we begin describing the implementation of their
confluence -- a staged abstract interpreter. We present a principled approach to
derive staged abstract interpreter from its unstaged version. One guiding
principle of our approach is that the code of the abstract semantics and the
code that optimizes should be separated. This it is an advantage of using
staging for abstract interpreters: the designer of the analysis has no need to
rewrite the analysis, and the performance improvement comes almost for free,
without any sacrifice of soundness or precision. Unsurprisingly, the staged
abstract interpreter we present in this section has the same abstract semantics
as the unstaged version we presented in Section~\ref{unstaged_abs}.

\subsection{Staged Lattices}

In Section~\ref{stagedpoly_lat}, we exploited the higher-kinded type @R@ to
leave space for staging lattices, now we instantiate the type @R@ to @Rep@ and
still use powersets as an example to present its staged version.

\begin{lstlisting}
trait RepLattice[A] extends Lattice[A, Rep]
implicit def RepSetLattice[T:Typ]: RepLattice[Set[T]] = 
  new RepLattice[Set[T]] {
    lazy val bot: Rep[Set[T]] = Set[T]()
    lazy val top: Rep[Set[T]] = throw new NotImplementedError()
    def $\sqsubseteq$(l1: Rep[Set[T]], l2: Rep[Set[T]]): 
      Rep[Boolean] = l1 subsetOf l2
    def $\sqcup$(l1: Rep[Set[T]], l2: Rep[Set[T]]): 
      Rep[Set[T]] = l1 union l2
    def $\sqcap$(l1: Rep[Set[T]], l2: Rep[Set[T]]): 
      Rep[Set[T]] = l1 intersect l2
  }
\end{lstlisting}

The type parameter @T:Typ@ of @RepLattice@ requires that the elements of sets
can also be staged. Otherwise, without knowing how to stage the elements in the
set, we can not stage the set either. The methods defined operate on type
@Rep[Set[T]]@, thus the underlying implementation such as @union@ and
@intersect@ will be mapped to a node in the IR graph during staging and
eventually emitted in the generated code. Again, other structures such as maps
and tuples are implemented in a similar way.

\subsection{Staged Abstract Semantics}

We have seen how to obtain a staged concrete semantics based on types, now we
take the same approach to obtain a staged abstract semantics. The basic
operations are largely kept the same as in the unstaged version, except the
types are changed to @Rep@. Besides, when we update the environment, the
identifier @x@ is known statically, but the environment map has type
@Rep[Map[Ident,Addr]]@, so we apply @lift@ to @x@ to turn it as a next-stage
value.

\begin{lstlisting}
trait RepAbsInterpOps extends Abstract with LMSOps {
  type R[+T] = Rep[T]
  val $\rho$0: Rep[Env] = Map[Ident, Addr]()
  val $\sigma$0: Rep[Store] = Map[Addr, Value]()
  def get($\rho$: Rep[Env], x: Ident): Rep[Addr] = $\rho$(x)
  def put($\rho$: Rep[Env], x: Ident, a: Rep[Addr]): 
    Rep[Env] = $\rho$ + (lift(x) -> a)
  def get($\sigma$: Rep[Store], a: Rep[Addr]): Rep[Value] = 
    $\sigma$.getOrElse(a, RepLattice[Value].bot)
  def put($\sigma$: Rep[Store], a: Rep[Addr], v: Rep[Value]): 
    Rep[Store] = $\sigma$ + (a -> RepLattice[Value].$\sqcup$(v, get($\sigma$, a)))
  def alloc($\sigma$: Rep[Store], x: Ident): Rep[Addr] = Addr(x)
  def num(i: Lit): Rep[Value] = Set(NumV())
  def branch0(cnd: Rep[Value], thn: => Ans, els: => Ans): Ans =
    thn $\sqcup$ els
  def prim_eval(op: Symbol, 
                v1: Rep[Value], v2: Rep[Value]): Rep[Value] = 
    Set(NumV())
  ...
}
\end{lstlisting}

Once more, the way we handle closures is the same as in the staged concrete
interpreter: the recursive call to @ev@ with the body expression @e@ is compiled
and specialized, the wrapper function @f@ will be a field value in a
@CompiledClo@ object and be generated for the next stage. At the end, we return
a singleton set:

\begin{lstlisting}
def close(ev: EvalFun)($\lambda$: Lam, $\rho$: Rep[Env]): Rep[Value] = {
  val Lam(x, e) = $\lambda$
  val f: Rep[(Value, Store)]=>Rep[(Value,Store)] = {
    case (args: Rep[Value], $\sigma$: Rep[Store]) =>
      val args = as._1; val $\sigma$ = as._2; val $\alpha$ = alloc($\sigma$, x)
      ev(e, put($\rho$, x, $\alpha$), put($\sigma$, $\alpha$, args))
    }
  Set[AbsValue](CompiledClo(fun(f)))
}
def apply_closure(ev: EvalFun)
  (f: Rep[Value], arg: Rep[Value], $\sigma$: Rep[Store]): Ans = {
    reflectEffect(ApplyClosure(f, arg, $\sigma$))
  }
\end{lstlisting}

When generating code for an application, we can not directly apply the callee.
Instead, we emit code that calls a next-stage function @apply_closures_norep@.
As its unstaged counterpart, function @apply_closures_norep@
non-deterministically applies multiple target closures with the argument and
latest store, and finally returns the joined value and a single store. We
provide the definition of @apply_closures_norep@ in the runtime supporting code.

\begin{lstlisting}
case ApplyClosures(fs, arg, $\sigma$) =>
  emitValDef(sym, "apply_closures_norep(" + 
                  quote(fs) + "," + quote(arg) + 
                  "," + quote($\sigma$) + ")")
\end{lstlisting}

\subsection{Staged Fixpoint Iteration} 

Our fixed-point iteration again relies on two caches @in@ and @out@, but the
iteration no longer be done at the current stage. Because the @in@ and @out@ are
both next-stage values, the test of whether @in@ and @out@ are equal is a
generated expression in the next stage, and we can only know the comparison
result at the next stage. In other words, we do not know how many iterations we
need to reach the fixed-point. To achieve this, we need to stage a function
value --- @iter_aux@ is generated as a recursive function of type @Rep[Unit => (Value,Store)]@ 
and will be invoked at the next stage.

\begin{lstlisting}
def iter(e: Expr, $\rho$: Rep[Env], $\sigma$: Rep[Store]): 
Rep[(Value,Store)] = {
  def iter_aux: Rep[Unit => (Value,Store)] = fun { () =>
    in = out; out = Map[Config, (Value,Store)]()
    cached_ev(e, $\rho$, $\sigma$)
    if (in === out) out((unit(e), $\rho$, $\sigma$)) else iter_aux()
  }
  iter_aux() // generated code that invokes iter_aux()
}
\end{lstlisting}

However, the instrumented evaluation function that uses the @in@ cache and
updates the @out@ cache can be completely eliminated by staging. Each recursive
call to @cached_ev@ will also be specialized if it is applied on subexpressions
of the analyzed program.

\begin{lstlisting}
def cached_ev(e: Expr, $\rho$: Rep[Env], $\sigma$: Rep[Store]): 
Rep[(Value, Store)] = {
  val cfg: Rep[Config] = (unit(e), $\rho$, $\sigma$)
  if (out.contains(cfg)) { out(cfg) }
  else {
    val ans0 = in.getOrElse(cfg, RepLattice[(Value, Store)].bot)
    out = out + (cfg -> ans0)
    val ans1 = evev(cached_ev)(e, $\rho$, $\sigma$)
    out = out + (cfg -> (ans0 $\sqcup$ ans1)); ans1
  }
}
\end{lstlisting}

\subsection{Specialized Data Structures} \label{staged_ds}

Now we have already obtained an end-to-end staged abstract interpreter that is
able to specialize an analysis. However, we treat the data structures such as
@Map@s as black-boxes, which means any operations on a @Map@ become code in the
next stage. But, as we identified when introducing the generic interface, the
keys of any environment maps are identifiers in the program, which are
completely known statically. This leaves us a chance to further specialize the
data structures. Assume the @Map[K,V]@ is implemented as a hash map, if the keys
$K$ are known, then the indices can be computed statically. Thus the specialized
map would be an array @Array[Rep[V]]@ whose elements are next-stage values; all
the accesses to the array is determined during staging.

Particularly, if we are specializing a monovariant analysis, the address space
is equivalent to the identifiers, then the environment component can be entirely
eliminated, and the store is a specialized map as array of @Rep[Value]@
elements.

\subsection{Modular Analysis for Free}

One of the challenges of modern static analysis is program usually depends on
large libraries programs~\cite{toman_et_al:LIPIcs:2017:7121}. Can we analyze
programs and libraries separately and reuse the result without losing precision?
Then we can reduce part of the overhead of repeatedly analyzing libraries for
different programs. Indeed, some static analyzers compute summary for a function
or a module, which can be reused later, however they are mostly too conservative
or unsound, which both lead to imprecision.

The specialization of abstract interpreter provides a chance to obtain such
partial analysis result in a mechanized way, but still keeps the analysis sound.
As we see when compiling the closures, we can specialize the abstract
interpreter with respect the body expression of the lambda term without knowing
the actual argument. The programs with some unknown variables are open programs,
which is exactly the case if we want to analyze programs in a modular way.

For a concrete analysis, for example, $k$-CFA ($k$ > 0) is naturally a
whole-program analysis, because it is inter-procedural and needs the last $k$
calling contexts to distinguish different call sites, where the calling contexts
are dynamic values. However, it is possible to analyze programs (libraries)
separately through specializing an abstract interpreter that generates the
analysis as the next-stage program and leave the unavailable programs and
calling contexts as dynamic parameters, and then install these contexts when we
have the whole program.

\iffalse
Another perspective: programs are data for an abstract interpreter, so if we
have $n$ programs, then maybe there can be $n$ stages. Probably we can analyze
first $m$ programs, and generate a residual abstract interpreter waiting for the
rest $(n-m)$ programs. These $(n-m)$ programs might be (abstract) arguments for
the first $n$ programs, and the abstract interpreter itself might be a partial
abstract interpreter.
\fi

\subsection{Discussion}

In the literature of partial evaluation, \citeauthor{10.1007/3-540-61580-6_11}
provided guidelines on what to do and not to do when specializing a concrete
interpreter \cite{10.1007/3-540-61580-6_11}. We borrow these guidelines and
extend them to abstract interpreters. We discuss decisions we made to achieve
this and examine some alternatives.

\paragraph{Big-step vs Small-step}

What we implemented is a big-step, compositional abstract interpreter, where
"compositional" means that every recursive call of our abstract interpreter is
applied to proper substructures of the current syntactic parameters
\cite{10.1007/3-540-61580-6_11}. This compositionality ensures that
specialization can be done by unfolding, as well as that the specialization
procedure terminates. Nevertheless, it is also possible to specialize a
small-step operational abstract semantics ---
\citeauthor{Johnson:2013:OAA:2500365.2500604} showed this in abstract
compilation \cite{Boucher:1996:ACN:647473.727587} style as one of their
optimizations of Abstract Abstract Machines
\cite{Johnson:2013:OAA:2500365.2500604}. However, the generated abstract
bytecode still requires another small-step abstract machine to execute, which is
an additional engineering efforts. Another alternative approach for efficient
specialization is to write the abstract interpreter in a big-step, monadic style
\cite{DBLP:journals/pacmpl/DaraisLNH17}.

% direct-style vs CPS

\paragraph{Correctness and Soundness}

Based on the assumption that LMS preserves the semantics during staging, we are
confident that the staged abstract interpreter does the same analysis compared
with the unstaged one. Moreover, the optimization done by staging does not
compromise any soundness.

