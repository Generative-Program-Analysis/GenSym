\section{Related Work}

\paragraph{Optimizing Static Analysis Through Specialization}
The underlying idea in this paper is closely related to abstract
compilation (AC) \cite{Boucher:1996:ACN:647473.727587}: removing the
interpretation overhead on traversing the syntax tree by
specialization. Specifically, the residual program of AC can be either
textual or closures. However, as we studied in Section \ref{cs_ac},
with the perspective from generative programming and monadic
interpreters, we make it more general, extensible and easy.
\citet{Johnson:2013:OAA:2500365.2500604} adapt the idea of closure
generation to optimize small-step abstract interpreter in
state-transition style. The analyzed program is firstly compiled to an
IR called "abstract bytecode", which are actually higher-order
functions and will be executed later on an abstract abstract machine
for that IR. \citet{damian1999partial} provides a formal treatment to
abstract compilation and Shiver's CFA, as well as proofs to establish
the correctness of the certified specialized analyzer.
\citet{amtoft1999partial} applied partial evaluation for
constraint-based control flow analysis.

Splitting an analysis into multiple stages is also studied for
analysis other than abstract-interpretation-based control-flow
analysis, though the formulation may very
different. \citet{DBLP:conf/cgo/HardekopfL11} apply staging to a
flow-sensitive pointer analysis. The first stage is to analyze the
program code to obtain a sparse representation, and then the second
stage conducts the flow-sensitive analysis by refining the first
one. Another large body of static analyses are implemented by
generating constraints firstly and then let a solver discharge these
proof obligations or find the fixed-point. Notable examples include
SAT- and SMT-based program analysis
\cite{Gulwani:2008:PAC:1375581.1375616} and Datalog-based points-to
analysis \cite{Smaragdakis:2015:PA:2802194.2802195}.
Similar to the idea in this paper, such analyses can be splitted to
staged.  One of the future work is to explore that whether these
analyses can be implemented and optimized from
metaprogramming. Souffl{\'e}, one of the state-of-the-art Datalog
engine that is commonly used in points-to analysis
\cite{Antoniadis:2017:PDS:3088515.3088522}, uses the idea of Futamura
projection for efficent implemetation \cite{10.1007/978-3-319-41540-6_23}.
But it only specializes Datalog rules to the running engine, instead
of a full pipeline from program AST to generating constraints and
solving constraints.

Recent work shows that the numerical abstract domains can be
specialized w.r.t. the static program structure, although not in the
way of using staging. For example, decomposing polyhedras
\cite{DBLP:conf/popl/SinghPV17, Singh:2017:PCD:3177123.3158143} to
smaller ones that depend on the variables involved in a statement can
significantly improve the running time and space of abstract
transformers.

\paragraph{Construction of Abstract Interpreters} Abstract interpretation
was proposed as a semantic-based approach to build sound static analysis by
approximation \cite{DBLP:conf/popl/CousotC77}. As to build semantic
artifacts, the Abstracting Abstract Machines (AAM)
\cite{DBLP:journals/jfp/HornM12, DBLP:conf/icfp/HornM10} approach
shows that abstract interpreters can be derived systematically from
concrete semantic artifacts. The AAM approach is closely related to
control-flow analysis for higher-order languages
\cite{Midtgaard:2012:CAF:2187671.2187672}.  Using monads to construct
abstract interpreters is explored by
\citet{Sergey:2013:MAI:2491956.2491979} and
\citet{DBLP:journals/pacmpl/DaraisLNH17,
  Darais:2015:GTM:2814270.2814308}.  The unstaged abstract interpreter
in this paper follows the abstracting definitional interpreters
approach from \citet{DBLP:journals/pacmpl/DaraisLNH17}.  Similar to
definitional abstract interpreters,
\citet{Wei:2018:RAA:3243631.3236800} reconstruct big-step abstract
interpreters with delimited continuations;
\citet{Keidel:2018:CSP:3243631.3236767} present a modular arrow-based
abstract interpreter that makes proof of soundness can be constructed
compositionally.

\citet{DBLP:conf/cc/CousotC02} proposed an abstract interpretation
framework for modular analysis. We borrow the notation of modular
analysis, \todo{but intentional properties are different, compare with
  summary}

\paragraph{Two-level Semantics} The idea of reinterpreting the semantics as
abstract interpretation can be traced to Nielson's two-level semantics
\cite{NIELSON1989117}; using two-level semantics for code generation
was also explored by \citet{NIELSON198859}.
\citet{Sergey:2013:MAI:2491956.2491979}'s work of monadic abstract
interpreters is also closely related to the two-level semantics: the
use of a generic interface with monads and then reinterpret it by
different semantics is already two-level. Instead of focusing on
semantics, this paper shows how a staged analyzer can be built and
used to increase efficieny of static analysis, we augment the monadic
abstract interpreter from another dimension, using code generation to
produce efficient low-level code. Hence the presented work can be
considered as an two-dimension application of the two-level semantics.

\paragraph{Partial Evaluation and Multi-stage Programming}
Partial evaluation as an automatic technique for program
specialization was studied comprehensively by
\citet{10.1007/3-540-61580-6_11, DBLP:books/daglib/0072559}.  In this
paper, we use mutli-stage programming as an approach to program
specialization. The Lightweight Modular Staging framework
\cite{DBLP:conf/gpce/RompfO10} we used in the paper relies on
type-level stage annotations.  Other notable implementations of MSP
exist in ML family, e.g., MetaML \cite{DBLP:conf/pepm/TahaS97} and
MetaOCaml \cite{DBLP:conf/gpce/CalcagnoTHL03,
  DBLP:conf/flops/Kiselyov14}.  The idea of staging an abstract
interpreter we presented in this paper is also applicable to other MSP
systems.  Multi-stage programming has been widely used to improve the
performance in many domains, such as optimizing compilers or
domain-specific languages \cite{DBLP:conf/pldi/RompfSBLCO14,
  DBLP:conf/snapl/RompfBLSJAOSKDK15,
  DBLP:journals/tecs/SujeethBLRCOO14, DBLP:conf/gpce/SujeethGBLROO13,
  DBLP:journals/jfp/CaretteKS09}, numerical computation \cite{PGL-038,
  DBLP:conf/pepm/AktemurKKS13}, generic programming
\cite{DBLP:journals/pacmpl/Yallop17,
  Ofenbeck:2017:SGP:3136040.3136060}, data processing
\cite{DBLP:conf/oopsla/JonnalageddaCSRO14,
  DBLP:conf/popl/KiselyovBPS17}, query compilation in databases
\cite{DBLP:conf/osdi/EssertelTDBOR18, DBLP:conf/sigmod/TahboubER18},
etc.

As an source of inspiration of this paper, Futamura projections
reveals a hierarchy and close relation between interpreters and
compilers, which was originally proposed by
\citeauthor{futamura1971partial} in 1970s\cite{futamura1971partial},
and later reprinted \cite{Futamura1999}.
\citeauthor{Amin:2017:CTI:3177123.3158140} consider a tower of
concrete interpreters and how to collapse them by using MSP -- it
would be interesting to explore this idea for multiple layers of
abstract interpreters \cite{Cousot:2019:AAI:3302515.3290355,
  Giacobazzi:2015:APA:2676726.2676987}.  \citet{10.1007/11561347_18}
discussed combining multi-stage programming with functors and
monads. But they didn't \todo{monad transfomers?}  Similar to the idea
in this paper, specializing monadic interpreters was explored by
\citet{DBLP:conf/dsl/SheardBP99, danvy1991compiling}. \todo{what's the
  difference?}

\paragraph{Control-flow Analysis} useful applications, why important
