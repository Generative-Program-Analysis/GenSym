\section{Evaluation} \label{evaluation}

To evaluate whether the performance can be improved through staging, we use the
abstract interpreter demonstrated in the paper to analyze some Scheme programs.
We first implement a desugaring transformation for a large subset of Scheme that
transforms into the small language we used in the paper. We use the same generic
interface with a 0-CFA-like abstraction, and the unstage abstract interpreter
forms the baseline.

All of our evaluation benchmarks were performed on an Ubuntu 16.04 LTS (Linux
kernel 4.4.0) machine with 4 Intel Xeon Platinum 8168 CPU at 2.7GHz and 3 TiB of
RAM. Although the machine has 96 cores and 192 threads in total, the abstract
interpreters only use one thread to run all the benchmark programs.

Our evaluations show that the staged abstract interpreter performs well in some
cases that are considered to be the worst scenarios for traditional $k$-CFA. The
staged version outperforms the unstaged abstract interpreter by a significant
margin (up to ~230\% performance gain) on these difficult ones.

\subsection{Benchmarks}

The benchmark programs we used are collected from previous papers
\cite{Johnson:2013:OAA:2500365.2500604, ashley:practical,
DBLP:journals/corr/abs-1102-3676} and exisiting artifacts
\footnote{https://github.com/ilyasergey/reachability}.

We used the following benchmark programs:

\begin{itemize}
    \item \textbf{fib}: Calculates the $n$-th fibonacci number.
    \item \textbf{rsa}: The RSA public key encryption algorithm.
    \item \textbf{kcfa3}: A difficult benchmark for $k$-CFA.
    \item \textbf{church}: Church numerals with additions and mutiplications.
    \item \textbf{fermat}: Fermat and Solovay-Strassen primality testing.
    \item \textbf{kcfa-worst-case-}\textit{n}: 
      Benchmark programs that are supposed to be tough cases for $k$-CFA; 
      the number $n$ indicates the depth of nesting lambda terms.
\end{itemize}

\subsection{Performance}

Considering that our abstract interpreters are implemented in Scala which will be 
affected by the JIT warm up times, we run all experiments 10 times and report
the statistical median values of the running times.

\todo{remove fib; report LOC; mean}

\begin{figure}[h]
\begin{tabular}{@{}llll@{}}
\toprule
    Program            & unstaged   & staged     & $\Delta$ of median \\ \midrule
    fib                & 1.020      & 2.046      & -50.14\%          \\
    rsa                & 35.928     & 145.306    & -75.27\%          \\
    kcfa3              & 2.934      & 5.671      & -48.27\%          \\
    church             & 884.800    & 545.736    & +62.13\%          \\
    fermat             & 15.673     & 68.790     & -77.22\%          \\
    kcfa-worst-16      & 320.689    & 301.564    & +6.34\%           \\
    kcfa-worst-32      & 3,685.615  & 1,854.461  & +98.74\%          \\
    kcfa-worst-64      & 49,422.666 & 14,849.725 & +232.82\%         \\
    \bottomrule
\end{tabular}
\caption{Evaluation result.} \label{evaluation_result}
\end{figure}

Figure~\ref{evaluation_result} shows the benchmark evaluation results. The
"unstaged" and "staged" columns show the median time to finish the analysis (in
milliseconds). The "$\Delta$ of median" column shows the performance change from
the unstaged abstract interpreter to the staged version.

As we can see from the table, the staged version significantly outperforms the
unstaged especially for the difficult ones. However, for some benchmarks, we
observe degraded performance. We conjecture this happens due to the fact that
the large size of the generated code is unoptimized -- as what we test here is a
simple implementation almost identical to what we described in the previous
sections, we believe the performance still can be improved if more engineering
effort is taken, such as optimizing code generation and avoiding code
duplication.

