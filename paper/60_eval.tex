\section{Evaluation} \label{evaluation}

We have designed, implemented and evaluated a prototype abstract interpreter for Scheme.
In order to compare results, our prototype includes both staged and unstaged versions of
the abstract interpreter doing the same control flow analysis.

Our evaluations have shown that the staged abstract interpreter performs well in general.
Particularly in some interesting cases that are considered to be the worst case scenario
in the control flow analysis, the staged version outperforms the unstaged abstract 
interpreter by a wide margin. 

However, there also exists some programs where the unstaged abstract interpreter yields 
results faster than the unstaged interpreter.

\subsection{Benchmarks}

Our evaluation is based on a suite of benchmark tests. Considering that our abstract 
interpreters are implemented in Scala which will be affected by JVM warm up times, programs
tend to run faster after a number of runs. In order to minimize the influence of JVM, 
we ran all experiments 10 times and take the statistical mean and median values of 
the running times.

%on a suite of benchmark scheme programs that are also used in \cite{Johnson:2013:OAA:2500365.2500604, ashley:practical}. 
We used the following benchmark programs:
\begin{itemize}
    \item \textbf{fib:} a program that recursively calculates the $n$-th fibonacci number in
        exponential time.
    \item \textbf{church:} a program introduced by \todo{cite Dimitrios Vardoulakis and Olin Shivers. CFA2: a Context-Free approach
        to Control-Flow analysis. Logical Methods in Computer Science,
        7(2), 2011.} to test distributivity of multiplication for numbers in Church Encoding \todo{ cite }
\end{itemize}

\subsection{Performance}
All of our evaluation benchmarks were performed on an Ubuntu 16.04 LTS (Linux 4.4.0) machine 
with 4 Intel(R) Xeon(R) Platinum 8168 CPUs running at 2.7GHz with a total of 3 TiB of RAM.
Although we were using an environment with 96 cores and 192 threads, we used only
one single thread for all the benchmark programs we run.

For each benchmark program, we ran our prototype abstract interpreters, both unstaged and staged.
\todo{Figure show data.}


\subsubsection{Baselines}
Small-step AAM

Big-step ADI

OAAM with AC \cite{Boucher:1996:ACN:647473.727587, Johnson:2013:OAA:2500365.2500604}

Benchmarks and other reference implementations: \url{https://github.com/ilyasergey/reachability}.

\subsection{Compositionality}

