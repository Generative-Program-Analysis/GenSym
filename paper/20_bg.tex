\newcommand{\TLang}{$L_\lambda$}

\section{Background}

In this section, we first described the language we will model and analyze, and present a unstaged concrete 
interpreter for it. Then, starting from this unstaged concrete interpreter, we proceed in two directions:
by abstracting it we obtain an unstaged abstract interpreter; by staging it we obtain a staged concrete
interpreter.

\subsection{The Target Language} \label{bg_lang}

As higher-order functions had been widely introduced in mainstream programming languages \todo{cite}, in this paper, 
we target a tiny higher-order functional language \TLang, which is based on direct-style lambda-calculus, 
and with the support of numbers, arithmetics, recursions and conditionals.
In section \ref{cases_imp}, we will see adding more imperative features to the language.
Since we are mostly interested in analyzing the dynamic behavior of the program, the static semantics and
type system will be omitted. We also assume all the input programs are well-typed.
The abstract syntax is described below:

\begin{lstlisting}
abstract class Expr
case class Lit(i: Int)
case class Var(x: String)
case class Lam(x: String, e: Expr)
case class App(e1: Expr, e2: Expr)
case class AOp(op: String, e1: Expr, e2: Expr)
case class Rec(x: String, e: Expr, body: Expr)
case class If0(cnd: Expr, thn: Expr, els: Expr)
\end{lstlisting}

We give the concrete semantics of \TLang by showing a big-step definitional interpreter for it.
But before doing that, let's first see a general interface for interpreters, whether concrete or abstract, 
staged or unstaged. The trait \texttt{Semantics} first declares several abstract type members, such as \texttt{Ident},
\texttt{Addr} and etc., which will be instantiated concretely later. The abstract types \texttt{Env} and \texttt{Store}
are required to conform their specifications, \texttt{EnvSpec} and \texttt{StoreSpec}, which define 
the operations such as lookup and update on them, nevertheless the detail of them are omitted.  
The store component is necessary when modeling side effects such as mutations. 
The trait also requires the implementators provide the initial values of environment and store.

%TODO: talk about alloc here?

\begin{lstlisting}
trait Semantics[R[_]] {
  type Ident;    type Addr;   type Value;    type Ans
  type Env <: EnvSpec[R, Ident, Addr, Env]
  type Store <: StoreSpec[R, Addr, Value, Store]
  val $\rho$0: Env;     val $\sigma$0: Store
  def alloc(x: Ident, $\sigma$: Store): R[Addr]
  type EvalFun = (Expr, Env, Store) => Ans
  def eval(ev: EvalFun)(e: Expr, $\rho$: Env, $\sigma$: Store): Ans
  def fix(ev: EvalFun => EvalFun): EvalFun = 
    (e, $\rho$, $\sigma$) =>  ev(fix(ev))(e, $\rho$, $\sigma$)
  def eval_top(e: Expr): Ans = fix(eval)(e, $\rho$0, $\sigma$0)
}
\end{lstlisting}

The evaluator \texttt{eval} is a open-recursive function that takes one and three arguments: the first
one \texttt{ev} represents the recursive function itself; the rest three are the AST representing program 
will be executed, the environment that maps from identifiers to addresses, and the store that maps addresses to values.
The interpreters that take control, environment and store components also can be obtained by 
refunctionalizing \cite{DBLP:conf/ppdp/AgerBDM03, Wei:2018:RAA:3243631.3236800} a small-step CESK 
machine \cite{DBLP:conf/popl/FelleisenF87}.
We provide a function \texttt{fix} to close the recursion by providing the result of \texttt{fix(ev)} 
to argument \texttt{ev}. Finally, a default implementation for the top-level evaluation function \texttt{eval\_top} is defined.
As we will see later, the open-recursion style is beneficial when implementing abstract interpreter and 
staged abstract interpreter \todo{cite adi}.

Readers may notice an additional notation \texttt{[R[\_]]} afterward the trait definition,
which indicates \texttt{Semantics} takes a type parameter \texttt{R}, where \texttt{R} represents
a higher-kinded type and will used to specify the binding-time, i.e., staged or not. 
For the moment, \texttt{R} will be instantiated unstaged --- an identity type constructor: 
\texttt{type NoRep[A] = A}. \todo{where talk about stage poly?}

Now we can instantiate a concrete interpreter. We first concretize the abstract types as follows.
Since the interpreter is concrete, the address space is infinite thus type \texttt{Int} is used here.
The final returned type \texttt{Ans} is a pair of \texttt{Value} and \texttt{Store}, where
the values are either closure or numbers.

\begin{lstlisting}
trait Concrete[R[_]] extends Semantics[R] {
  type Ident = String;         type Addr = Int
  type Ans = (Value, Store);   abstract class Value
  case class CloV($\lambda$: Lam, $\rho$: Env) extends Value
  case class NumV(i: Int) extends Value
}
\end{lstlisting}

Then we define evaluation function that dispatches differently for different expression.
Note that an allocation function is used when we need to update the store: 
in the concrete case, function \texttt{alloc} always returns a fresh address of the store.

\begin{lstlisting}
object NoRepConcreteEval extends Concrete[NoRep] {
  def alloc(x: Ident, $\sigma$: Store): Addr = $\sigma$.size + 1
  def eval(ev: EvalFun)(e: Expr, $\rho$: Env, $\sigma$: Store): Ans = 
    e match {
      case Lit(i) => (NumV(i), $\sigma$)
      case Var(x) => ($\sigma$($\rho$(x)), $\sigma$)
      case Lam(x, e) => (CloV(Lam(x,e), $\rho$), $\sigma$)
      case App(e1, e2) =>
        val (CloV(Lam(x,e), $\lambda\rho$), e1$\sigma$) = ev(e1, $\rho$, $\sigma$)
        val (e2v, e2$\sigma$) = ev(e2, $\rho$, e1$\sigma$)
        val $\alpha$ = alloc(x, e2$\sigma$)
        ev(e, $\lambda\rho$ + (x -> $\alpha$), e2$\sigma$ + ($\alpha$ -> e2v))
      case If0(cnd, thn, els) => ...
      case AOp(op, e1, e2) => ...
      case Rec(x, e, body) => ...
    }
}
\end{lstlisting}

\subsection{From Interpreter to Abstract Interpreter}

\begin{lstlisting}
trait Abstract[R[_]] extends Semantics[R] {
  type Ident = String;         case class Addr(x: Ident)
  type Ans = (Value, Store);   type Value = Set[AbsValue]
  abstract class AbsValue
  case class CloV($\lambda$: Lam, $\rho$: Env) extends AbsValue
  case class NumV() extends AbsValue
}
\end{lstlisting}

Address just record the variable name, more sensitivity in section ...

other big-step abstract interpreter, 
big-step abstract interpreter using delimited control operators \cite{Wei:2018:RAA:3243631.3236800},
monadic big-step abstract interpreter \cite{DBLP:journals/pacmpl/DaraisLNH17},
arrow-based abstract interpreter \cite{Keidel:2018:CSP:3243631.3236767}.

\subsection{From Interpreter to Staged Interpreter}

\subsubsection{Multi-Stage Programming}
\todo{example, power?}
general idea; LMS framework in scala. MetaOCaml?

Other notable implementations of MSP exist in ML family, for examples, MetaML \cite{DBLP:conf/pepm/TahaS97} 
and MetaOCaml \cite{DBLP:conf/gpce/CalcagnoTHL03, DBLP:conf/flops/Kiselyov14}.
Compared with the LMS approach in Scala, MetaML/MetaOCaml use term-level annotations such as brackets, 
escape, and run. Notwithstanding, we use LMS in this paper, the idea of staging an abstract interpreter
still applies with other MSP implementations.

\subsubsection{Staged Interpreters} 
