\section{Case Study}

\subsection{Abstract Compilation a la Staging}

\citeauthor{Boucher:1996:ACN:647473.727587} introduced abstract compilation as a new
implementation technique for abstract interpretation based static analysis \cite{Boucher:1996:ACN:647473.727587}.
Their work is esentially a partial evaluated CE (\textit{Control-Environment}) abstract
interpreter written in a big-step semantics for programs written in continuation-passing style.
The analysis it can perform is a monovariant control flow analysis, sepcifically, Shiver's original
formulation of 0-CFA \cite{Shivers:1991:SSC:115865.115884, Shivers:1988:CFA:53990.54007}.
After the partial evaluation on input program, the \textit{control} part of abstract interpreter
is disappeared, instead, the compiled analysis is a closure (higher-order function) and
only takes an environment as argument,
and the overhead of traversing the abstract syntax tree of source program also has been eliminated.
Since the target language is written a continuation-passing style, so there is
no explict evaluation context component when constructing the abstract interpreter
(rather than a CEK machine).

In this section, we firstly review their work of closure generation,
and reproduce their result by adding stage annotations to 0-CFA.
Show that \citeauthor{Boucher:1996:ACN:647473.727587}'s abstract compilation can be understood as an instance of staged 
abstract interpreter.

Advantages: (almost?) no need to modify the analysis code; staging the store data structure

(Fianally, we should briefly summarize the difference between them:
staging vs partial evaluation; code generation; engineering effort; performance)

(Specific -> General vs General vs Specific)

\subsection{Control Flow Analysis}

context/path/flow-sensitive

store-widening

using different $k$ for $k$-CFA

\subsection{Scale to Imperative Languages}

Assignments, Loops, Exceptions

Options: Numerical Analysis, Data-flow analysis, taint analysis
