\section{Introduction} \label{intro}

Statically analyzing semantic properties of a program is a widely
known undecidable class of problem. Abstract interpretation as a
lattice-based approach to construct sound static analysis was proposed
by \citet{DBLP:conf/popl/CousotC77}. With the Galois connection, one
can approximate to the program runtime behaviors by computing the
fixed points on the abstract domain. Despite the tremendous
theoretical development of abstract interpretation over the years, on
the side of pragmatics, constructing artifacts and analyzers that
perform sound abstract interpretation was considered abstruse and
complicated in practice for a long time.

In recent years, we observe a rich progress on how we can construct
abstract interpreters from systematic principles, instead of ad-hoc
engineerings. A notable one is the Abstracting Abstract Machine (AAM)
methodology \cite{DBLP:journals/jfp/HornM12, DBLP:conf/icfp/HornM10},
which uncovers an approach to derive sound abstract interpreters from
their concrete counterparts, where the soundness can be easily
established by examining the transformation of semantic artifacts. For
example, a CEK machine \cite{DBLP:conf/popl/FelleisenF87} for concrete
execution can be easily refactored to an effective 0-CFA control-flow
analysis \cite{Shivers:1988:CFA:53990.54007, Midtgaard:2012:CAF:2187671.2187672}
by first tweaking the environment dereference as a nondeterministic choice,
then allocating continuations in the environment, and constraining the
addresses space to be finite. This systematic abstraction approach can
be tailored to different analyses \todo{cite}
\cite{DBLP:conf/icfp/Gilray0M16, DBLP:conf/popl/GilrayL0MH16} and and
has been applied to multiple variants of small-step abstract machines
\cite{DBLP:journals/jfp/HornM12, DBLP:conf/icfp/HornM10,
  Sergey:2013:MAI:2491956.2491979} and big-step definitional
interpreters \cite{Wei:2018:RAA:3243631.3236800,
  DBLP:journals/pacmpl/DaraisLNH17, Keidel:2018:CSP:3243631.3236767}.

More pragmatically, as an implementation strategy, several purely
functional programming approaches to build abstract interpreters were
emerged, for examples, using monads or arrows. The pure approach
provides much benefits: 1) the abstract interpretation artifacts are
compositional and can be constructed modularly, e.g., by using
monad transformers. 2) The soundness of analysis can be proved more easily, by
mechnization \cite{Darais:2016:CGC:2951913.2951934} or paper-based
proof \cite{Keidel:2018:CSP:3243631.3236767}.  3) The correctness of
implementations can be reasoned by the programmers more condidently,
e.g., by leveraging equational reasoning.

However, besides the intrinsic complexity of static analysis, there are
additional abstraction penalties with these high-level programming approaches.
First, similar to concrete interpreters, the abstract interpreter analyzes the
program by traversing the abstract syntax tree, which poses the interpretive
overhead, e.g., pattern matching on the AST and recursive calls on the sub
expressions. Those kind of overhead can be negligible if the abstract
interpreter only runs on the program once, but also can be accumulated
significantly if it runs repeatedly, for example, on libraries. Second, the
abstract interpreter written in pure languages usually extensively uses effect
systems to model the behaviors of abstract interpretatation, such as
nondeterminism. For example, \citet{DBLP:journals/pacmpl/DaraisLNH17} and
\citet{Sergey:2013:MAI:2491956.2491979}'s monadic abstract interpreters use
monad transformers; the very recent \citet{Githubsemantic}'s \textsc{Semantic} framework
uses extensible effects. With certain merits and elegance of the implementation,
the effect systems may also introduce additional performance overhead.

In this paper, we propose an abstraction-without-regret approach to
eliminate those performance penalties for abstract interpreters
written in purely functional programming style. In short, our approach
borrows ideas from program specialization and high-performance DSLs,
and applies them to a particular kind of meta-programs, abstract
interpreters.
1) Using multi-stage programming (the Lightweight Modular Staging
framework \cite{DBLP:conf/gpce/RompfO10}, in this paper), we can
specialize the abstract interpreter with respect to an input program
and then generate efficient low-level code that does the actual
analysis. The specialization can also eliminate the effect layers in the generated code.
2) We use a tagless-final approach to embed the target
language interpreter in a high-level host language (Scala), allowing
the user to implement different semantics modularly, including the
abstract semantics and staging, without changing the main generic
interpreter. Together with the multi-stage programming and type-level
annotations, this design allows user to derive staged abstract
interpreters without intrusive changes to the unstaged one, thus the soundness is preserved. 
Therefore, our approach is no regret in the sense of both performance and least
engineering efforts to achieve such performance. We elaborate these
two main ideas in detail as follow.

\paragraph{Futumura Projection of Abstract Interpreters}

The idea of 

specialization of interpreters/abstract interpreters

\paragraph{Generic Interpreters}

specializing interpreters is not new, Futumura projections, four different tasks
for language designers and engineers, writing definitional interpreters for the
language (concrete semantics close to deno/op semantics), compilers for the
language (translate to another language, low-level), writing static analyzer for
the language (abstract semantics property we want to analyze), and a performant
static analyzer (ultimate goal of this paper). shared a single generic
interpreter

\paragraph{Applications}

\paragraph{Contribution}

\paragraph{Oragnization}

\iffalse
On the other side, static analysis is a tradeoff between performance and
precision: higher precision usually leads to longer running time.

4. Existing method to improve the performance is adhoc, engineering heavy, require to rewrite the optimized version, therefore harder to reason about the correctness
6. program analyzers are also meta-programs, they manipulate other programs as data objects
\fi
