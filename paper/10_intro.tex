\section{Introduction}

Futamura projections \cite{Futamura1999, futamura1971partial} reveal the close
connection between compilers and interpreters. The first Futamura projection
specifically shows that specializing an interpreter with respect to the input
program yields an equivalent executable. Partial evaluation
\cite{DBLP:books/daglib/0072559} was the first proposed approach to realize
Futamura projections, which first identifies the binding-time of variables in
the program, i.e., they can be known whether statically or dynamically, and then
evaluates the static part, and finally generates a residual program that solely
relies on the dynamic part. However, given an arbitrary program, precisely
analyzing its binding-time is hard in general. As an alternative and pragmatic
approach to specialization and partial evaluation, multi-stage programming (MSP
for short) \cite{taha1999multi, DBLP:conf/pepm/TahaS97} requires the stage
annotations (i.e., binding-time annotations) to be explicit from the programmers.
These staging annotations identify which part of the inputs should be
specialized. As a classical example, we consider the power function to introduce
the idea of MSP:

\begin{lstlisting}
def power(b: Rep[Int], e: Int): Int = 
  if (e == 0) 1 else e * power(b, e-1)
\end{lstlisting}

If the programmer identifies that @b@ will be known in the future stage (as shown
on its type @Rep[Int]@ -- a representation of @Int@) and @e@ is known in the
current stage, say 5, then we can specialize the function @power@ with
respect to @e = 5@ and generate a specialized function that runs faster:

\begin{lstlisting}
def power5(b: Int) = b * b * b * b * b
\end{lstlisting}

Roughly at the same time when Futamura projections was visioned in 1970s,
\citeauthor{DBLP:conf/popl/CousotC77} proposed abstract interpretation as a
lattice-based semantic approach to construct sound static analysis, by
approximation of fixpoints \cite{DBLP:conf/popl/CousotC77}.
Recent progress such as Abstracting Abstract Machines (AAM) uncovers a
methodology to derive sound abstract interpreters from their concrete
counterparts, and it has been applied to different variants of definitional
interpreters and abstract machines \cite{DBLP:journals/jfp/HornM12,
  DBLP:conf/icfp/HornM10, DBLP:journals/pacmpl/DaraisLNH17}.
Given the structural similarity between concrete and abstract interpreters,
intellectually it is natural to raise the question whether it is possible to
specialize an abstract interpreter, and how can we effectively do that having
considered their different functionalities.

This paper studies the confluence of two old ideas --- Futamura projection and
abstract interpretation, but from the perspective of their recent realizations
--- multi-stage programming and definitional abstract interpreters.
To be specific, we present the application of the first Futamura projection on
definitional abstract interpreters, which intersects as staged abstract interpreters.
By borrowing the ideas from \note{monadic interpreters} and embedded
domain-specific languages, we develop an approach to fulfill four different
semantics (Figure~\ref{confluence}) in a unified framework.

% FIXME: and the approach of constructing optimizing abstract interpreters by multi-stage programming.

\begin{figure}[h]
  \begin{tikzpicture}
  \matrix (m) [matrix of math nodes,row sep=3em,column sep=4em,minimum width=2em]
  {
    \begin{smallmatrix} \text{unstaged} \\ \text{abstract} \\ \text{interpreters} \end{smallmatrix} & \begin{smallmatrix} \text{staged} \\ \text{abstract} \\ \text{interpreters} \end{smallmatrix} \\
      \begin{smallmatrix} \text{unstaged} \\ \text{concrete} \\ \text{interpreters} \end{smallmatrix} & \begin{smallmatrix} \text{staged} \\ \text{concrete} \\ \text{interpreters} \end{smallmatrix} \\ };
  \path[-stealth]
    (m-1-1) edge node [above, font=\scriptsize] {$\updownarrows$} (m-1-2)
    (m-2-1) edge node [left, font=\scriptsize] {$\alpha$} (m-1-1)
    (m-2-1) edge node [below, font=\scriptsize] {$\updownarrows$} (m-2-2)
    (m-2-2) edge node [right, font=\scriptsize] {$\alpha$} (m-1-2);
  \end{tikzpicture}
  \caption{The confluence of staging and approximation.}
  \label{confluence}
\end{figure}

\note{besides the intellectual merit of the confluence diagram, practical
problem: can we speed up a static analyzer without pervasive changes of the
abstract semantics, even without changing a single line of the abstract
interpreter?}

\todo{Move some into related work}
Multi-stage programming has been widely used to improve the performance in many
domains, such as optimizing compilers or domain-specific languages
\cite{DBLP:conf/pldi/RompfSBLCO14, DBLP:conf/snapl/RompfBLSJAOSKDK15,
DBLP:journals/tecs/SujeethBLRCOO14, DBLP:conf/gpce/SujeethGBLROO13,
DBLP:journals/jfp/CaretteKS09}, numerical computation \cite{PGL-038,
DBLP:conf/pepm/AktemurKKS13}, generic programming
\cite{DBLP:journals/pacmpl/Yallop17, Ofenbeck:2017:SGP:3136040.3136060}, data
processing \cite{DBLP:conf/oopsla/JonnalageddaCSRO14,
DBLP:conf/popl/KiselyovBPS17}, query compilation in databases
\cite{DBLP:conf/osdi/EssertelTDBOR18, DBLP:conf/sigmod/TahboubER18}, etc.
Likewise, specializing static analyses by partial evaluation emerged in late 90s
\cite{damian1999partial, amtoft1999partial, Boucher:1996:ACN:647473.727587,
ashley:practical}, and indeed it is able to effectively remove the interpretive
overhead of repeatedly traversing the abstract syntax tree. However, these
previous works focused mostly on one particular analysis, or required to
completely rewrite the analyzer. Hence, it is worth to investigate the idea from
a modern perspective based on generative programming, especially for a general
abstract interpreter that models direct-style $\lambda$-calculus, imperative
features such as mutation, and different abstract domains. One technical
challenge is the binding-time engineering when non-determinism, fixed-point
iterations, different abstract domains and staging are introduced at the same
time. In this paper, we present an end-to-end design and implementation of
staging an abstract interpreter; that means not only the interpreter that
traverses the abstract syntax tree, but also the data structure of abstract
domains, abstract environment and heap, and the fixed-point iteration are
staged.

\todo{rewrite this paragraph}
On the other hand, the abstract interpreter, as a semantic artifact, should be
written in a style that is easy for people to communicate the formulation and
abstraction, but also can be implemented efficiently. As the slogan of
multi-stage programming said, "abstraction without regret", we draw connection
between high-level description and efficient implementation of abstract
interpreters, just like the connection between concrete interpreters and
compilers drawn by Futamura projection. Particularly, we show an easy but
systematic way of adding stage annotations to the abstract interpreter without
changing the code of the interpreter skeleton, which is shared between four concrete
or abstract, unstaged or staged interpreters. We use the LMS framework for staging,
which allows us only use types to annotate the binding-time. Therefore, the
proposed approach bridges the gap between designing sound static analysis and
implementing efficient program analyzer.

%Of course, all interpreters are metalinguistic abstractions, but some
%interpreters are more "abstract" than others \todo{maybe rephrase}.
%Particularly, we also would like a systematic approach to optimize program
%analyzers and meanwhile minimally modifying the analyzer programs.

\iffalse
When staging a concrete interpreter, the programmers need to distinguish static
and dynamic values --- the given program to be executed by the interpreter is
classified as static because it is known at compile-time, and the inputs to that
program are dynamic. However, when staging an abstract interpreter, this
distinction does not exist anymore. Because the abstract interpreter
instantiates all the inputs as some form of abstract values, which are usually
top elements in their abstract domains and are also statically known. Then what
is the point of staging if there is no such distinction? A surprising by-product
of thinking about this question is to realize that we can apply the staged
abstract interpreter on \textit{open} programs, and the free variables
representing other parts of the program (e.g., libraries) are dynamic inputs,
therefore we obtain a modular analysis through staging, mechanically.
\note{TR: I don't understand this. The program structure is static, the abstract values
are still dynamic, no? They change in every iteration of the fixpoint algorithm}
\note{TR: why is it a big deal. On Closed programs we obtain a constant factor,
but modular analysis has different asymptotics}
\fi

When targeting higher-order programs, either staged concrete interpreters or
staged abstract interpreters are able to compile a closure, i.e., specialize the
call of interpreter with respect to the body expression of the lambda term
without knowing the actual argument value. By generalizing this observation, we
can actually specialize the abstract interpreter with any open programs, which
unexpectedly leads to a modular analysis and improves scalability. An open
program contains free variables, which represent other parts of the program and
will be left as dynamic values. For instance, one challenge in static analysis
of modern software is that the programs are usually shipped with large library
code \cite{toman_et_al:LIPIcs:2017:7121}, for example, it has been shown that
analyzing a simple "Hello World" program in Java depends on additional 3,000
classes in the library \cite{DBLP:conf/oopsla/KulkarniMZN16}. A precise
whole-program analysis formulated as abstract interpreters can be very expensive
due to the scale of the program and the inherent complexity of the algorithm.
However, analyzing these libraries is sometimes unnecessarily repeated. When
applying the staged abstract interpreter on such open programs, we leave the
unknown arguments and calling contexts as dynamic values and generate partial
analyzing result which is represented as a residual program. The partial
analyzing result can be reused and composed with the analyzing result of
application code when available later. Therefore we mechanically obtain a
modular analysis through staging, even though the original algorithm is
formulated as a whole-program analysis.

\iffalse
It has been observed that partially applying context-sensitivity on selected
portion of the program could improve the precision and efficiency
\cite{zipper2018, Kastrinis:2013:HCP:2491956.2462191}. We show that staging
abstract interpreters as an approach to effectively implement hybird
context-sensitivity\todo{}.
\fi

\note{
  Expression problem \cite{DBLP:conf/icfp/GibbonsW14, DBLP:conf/ecoop/KrishnamurthiFF98, expproblem},
  deep embedding, four different interpretations.
}

\subsection{Contributions}

This paper makes the following contributions:

\begin{itemize}
  \item Starting from a generic definition of an interpreter for a higher-order
    functional language, whose syntax and semantics are described in
    Section~\ref{bg_lang}, we show its concrete interpreter
    (Section~\ref{unstaged_conc}), staged concrete interpreter
    (Section~\ref{stagedinterp}), abstract interpreter (Section~\ref{unstaged_abs})
    and staged abstract interpreter (Section~\ref{sai}). Our design allows to
    derive these four different kinds of interpreters by mechanically changing the
    underlying semantics or stage annotations.
    %We later extend the staged abstract interpreter to support imperative
    %language features (Section~\ref{cases_imp}), such as states and loops.
    
  \item As shown in Section~\ref{sai}, we demonstrate that if we apply the
    abstract interpreters on open programs, it not only improves the
    \textit{efficiency} but also the \textit{scalability}, which are two major
    issues in static analysis.
    
  \item To demonstrate the applicability and usefulness of the proposed
    approach, we conduct three case studies in Section~\ref{cases_study}:
    
    \subitem (1) We revisit Abstract Compilation \cite{Boucher:1996:ACN:647473.727587} in Section~\ref{cs_ac}, 
    which was originally a closure generation technique applied on 0-CFA.
    We show that abstract compilation can be understood and easily implemented
    as an instance of our staged abstract interpreter. But unlike abstract
    compilation, our approach does not require the implementers to rewrite the
    analysis.
    
    \subitem (2) We demonstrate pushdown control-flow analysis techniques
    (Section~\ref{cfa}), such as context-sensitive analysis and more precise
    stores can be integrated to staged abstract interpreters without losing any
    precision. 
    
    \subitem (3) We show that in an imperative language
    (Section~\ref{cases_imp}), by using an interval domain as abstract domain
    (as well as staged), the staged abstract interpreter performs numerical 
    analysis.
    
    % (Optional: taint/information flow analysis?)
  \item We empirically evaluate the performance improvement by staging a
    big-step abstract interpreter for control-flow analysis
    (Section~\ref{evaluation}). We compare our staged abstract interpreter with the
    unstaged version.
\end{itemize}
