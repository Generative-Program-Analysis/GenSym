It is well known that a staged interpreter is a compiler, which provides performance improvement
by specializing the interpreter to a given program.

In this paper, we study \textit{abstract} interpreters combined with multi-stage programming, i.e., the 
staged abstract interpreters. 
By staging the abstract interpreter with respect to a program, we obtain a specialized analysis that runs faster.
By appying the staged abstract interpreter with \textit{open} programs and considering the free variables as dynamic inputs, 
we obtain a modular analysis that generates sound partial analysis results which can be composed 
and reused later without losing precision, though the original abstract interpreter is a whole-program 
analysis algorithm.

Using the idea of staged abstract interpreters, we show several case studies, including 
\citeauthor{Boucher:1996:ACN:647473.727587}'s abstract compilation of Shiver's 0-CFA, pushdown
control-flow analysis with context/path/flow-sensitivity and store-widening, and a numerical
analysis on a imperative language.

We also empirically evaluate the improvement of performance on benchmark programs for a subset of Scheme.
The result shows ...

