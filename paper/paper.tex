%% For double-blind review submission, w/o CCS and ACM Reference (max submission space)
\documentclass[acmsmall,review,anonymous]{acmart}\settopmatter{printfolios=true,printccs=false,printacmref=false}
%% For double-blind review submission, w/ CCS and ACM Reference
%\documentclass[acmsmall,review,anonymous]{acmart}\settopmatter{printfolios=true}
%% For single-blind review submission, w/o CCS and ACM Reference (max submission space)
%\documentclass[acmsmall,review]{acmart}\settopmatter{printfolios=true,printccs=false,printacmref=false}
%% For single-blind review submission, w/ CCS and ACM Reference
%\documentclass[acmsmall,review]{acmart}\settopmatter{printfolios=true}
%% For final camera-ready submission, w/ required CCS and ACM Reference
%\documentclass[acmsmall]{acmart}\settopmatter{}


%% Journal information
%% Supplied to authors by publisher for camera-ready submission;
%% use defaults for review submission.
\acmJournal{PACMPL}
\acmVolume{1}
\acmNumber{ICFP} % CONF = POPL or ICFP or OOPSLA
\acmArticle{1}
\acmYear{2018}
\acmMonth{1}
\acmDOI{} % \acmDOI{10.1145/nnnnnnn.nnnnnnn}
\startPage{1}

%% Copyright information
%% Supplied to authors (based on authors' rights management selection;
%% see authors.acm.org) by publisher for camera-ready submission;
%% use 'none' for review submission.
\setcopyright{none}
%\setcopyright{acmcopyright}
%\setcopyright{acmlicensed}
%\setcopyright{rightsretained}
%\copyrightyear{2018}           %% If different from \acmYear

%% Bibliography style
\bibliographystyle{ACM-Reference-Format}
%% Citation style
%% Note: author/year citations are required for papers published as an
%% issue of PACMPL.
\citestyle{acmauthoryear}   %% For author/year citations


%%%%%%%%%%%%%%%%%%%%%%%%%%%%%%%%%%%%%%%%%%%%%%%%%%%%%%%%%%%%%%%%%%%%%%
%% Note: Authors migrating a paper from PACMPL format to traditional
%% SIGPLAN proceedings format must update the '\documentclass' and
%% topmatter commands above; see 'acmart-sigplanproc-template.tex'.
%%%%%%%%%%%%%%%%%%%%%%%%%%%%%%%%%%%%%%%%%%%%%%%%%%%%%%%%%%%%%%%%%%%%%%


%% Some recommended packages.
\usepackage{booktabs}   %% For formal tables:
                        %% http://ctan.org/pkg/booktabs
\usepackage{subcaption} %% For complex figures with subfigures/subcaptions
                        %% http://ctan.org/pkg/subcaption


\begin{document}

%% Title information
\title[Short Title]{Staged Abstract Interpreter}         %% [Short Title] is optional;
                                        %% when present, will be used in
                                        %% header instead of Full Title.
\titlenote{with title note}             %% \titlenote is optional;
                                        %% can be repeated if necessary;
                                        %% contents suppressed with 'anonymous'
\subtitle{Subtitle}                     %% \subtitle is optional
\subtitlenote{with subtitle note}       %% \subtitlenote is optional;
                                        %% can be repeated if necessary;
                                        %% contents suppressed with 'anonymous'


%% Author information
%% Contents and number of authors suppressed with 'anonymous'.
%% Each author should be introduced by \author, followed by
%% \authornote (optional), \orcid (optional), \affiliation, and
%% \email.
%% An author may have multiple affiliations and/or emails; repeat the
%% appropriate command.
%% Many elements are not rendered, but should be provided for metadata
%% extraction tools.

%% Author with single affiliation.
\author{First1 Last1}
\authornote{with author1 note}          %% \authornote is optional;
                                        %% can be repeated if necessary
\orcid{nnnn-nnnn-nnnn-nnnn}             %% \orcid is optional
\affiliation{
  \position{Position1}
  \department{Department1}              %% \department is recommended
  \institution{Institution1}            %% \institution is required
  \streetaddress{Street1 Address1}
  \city{City1}
  \state{State1}
  \postcode{Post-Code1}
  \country{Country1}                    %% \country is recommended
}
\email{first1.last1@inst1.edu}          %% \email is recommended

%% Author with two affiliations and emails.
\author{First2 Last2}
\authornote{with author2 note}          %% \authornote is optional;
                                        %% can be repeated if necessary
\orcid{nnnn-nnnn-nnnn-nnnn}             %% \orcid is optional
\affiliation{
  \position{Position2a}
  \department{Department2a}             %% \department is recommended
  \institution{Institution2a}           %% \institution is required
  \streetaddress{Street2a Address2a}
  \city{City2a}
  \state{State2a}
  \postcode{Post-Code2a}
  \country{Country2a}                   %% \country is recommended
}
\email{first2.last2@inst2a.com}         %% \email is recommended
\affiliation{
  \position{Position2b}
  \department{Department2b}             %% \department is recommended
  \institution{Institution2b}           %% \institution is required
  \streetaddress{Street3b Address2b}
  \city{City2b}
  \state{State2b}
  \postcode{Post-Code2b}
  \country{Country2b}                   %% \country is recommended
}
\email{first2.last2@inst2b.org}         %% \email is recommended


%% Abstract
%% Note: \begin{abstract}...\end{abstract} environment must come
%% before \maketitle command
\begin{abstract}
  We present Staged Abstract Interpreter
\end{abstract}

%% 2012 ACM Computing Classification System (CSS) concepts
%% Generate at 'http://dl.acm.org/ccs/ccs.cfm'.
\begin{CCSXML}
<ccs2012>
<concept>
<concept_id>10011007.10011006.10011008</concept_id>
<concept_desc>Software and its engineering~General programming languages</concept_desc>
<concept_significance>500</concept_significance>
</concept>
<concept>
<concept_id>10003456.10003457.10003521.10003525</concept_id>
<concept_desc>Social and professional topics~History of programming languages</concept_desc>
<concept_significance>300</concept_significance>
</concept>
</ccs2012>
\end{CCSXML}

\ccsdesc[500]{Software and its engineering~General programming languages}
\ccsdesc[300]{Social and professional topics~History of programming languages}
%% End of generated code


%% Keywords
%% comma separated list
\keywords{generic programming, abstract interpretation, static analysis}  %% \keywords are mandatory in final camera-ready submission


%% \maketitle
%% Note: \maketitle command must come after title commands, author
%% commands, abstract environment, Computing Classification System
%% environment and commands, and keywords command.
\maketitle

\section{Introduction}

\textbf{Outline(Proposed)}.

1. Revisit abstract compilation (generates textual program and closures) 

2. Generalize the idea to staged abstract interpreter.
We implement staged abstract interpreter within a big-step abstract semantics 
(abstract definitional interpreter) using Scala with LMS framework.
(ADI in Racket: \url{https://github.com/plum-umd/ abstracting-definitional-interpreters/}).
Show that abstract compilation can be understood as an instance of staged 
abstract interpreter.

Extension: can we do it with small-step semantics? or staging a transition system?

3. Improve performance (closure is still not an efficient enough representation).

Show it generates sound and high-performance low-level analysis code with 
various precisions and sensitivities, for instances, k-CFA, pushdown or m-CFA. 

Store/set operations (join, update) is also expensive, generating efficient store implementation\cite{liang2014fast}.

4. Modular analysis for free? 

Motivation: one of the challenges of modern static analysis is program usually depends on
large libraries programs\cite{toman_et_al:LIPIcs:2017:7121}. 
Can we analyze programs and libraries separately without losing precision? So that we can 
reduce part of the overhead of repeatedly analyzing libraries for different programs.
Similarly, some static analyzers compute summary for a function or a module, that can be reused
later (like Facebook Infer). But to my knowledge, they are mostly too conservative (context-insensitive) 
or unsound, which both lead to imprecision.

Application: for example, k-CFA (k > 0) is naturally a kind of whole program analysis,
because it is interprocedural and need the last k calling contexts to distinguish
different call sites.
But can we analyze programs (libraries) separately which generate the specialized 
analysis and leave the unavailable programs (for the moment) as dynamic parameters, 
and then install these contexts when we have the whole program.

Another perspective: programs are data for an abstract interpreter, so if we have $n$ programs, 
then maybe there can be $n$ stages. 
Probably we can analyze first $m$ programs, and generate a residual abstract interpreter
waiting for the rest $(n-m)$ programs.

Not sure how much improvement of performance.

5. Show the evaluation result on benchmark programs. 
Compare with normal abstract interpreter, 
abstract compilation\cite{Boucher:1996:ACN:647473.727587}, and OAAM \cite{Johnson:2013:OAA:2500365.2500604}.
Benchmarks and other reference implementations: \url{https://github.com/ilyasergey/reachability}.

\section{Background}

\subsection{Abstracting Abstract Machine}

\subsection{Staged Interpreter}

\section{Abstract Compilation, Revisited}

\section{Staged Abstract Interpreter}

\section{Evaluation}

\section{Related Work}

\textbf{Abstract Compilation}. The idea in this paper is closely inspired by 
abstract compilation \cite{Boucher:1996:ACN:647473.727587}.
\citeauthor{Boucher:1996:ACN:647473.727587} presented abstract compilation
techniques as an efficient implementation of the monovariant flow analysis 
\textit{0}-CFA for programs in continuation-passing style. The key idea is to 
reduce the interpretation overhead on traversing the syntax tree by partial evaluation.
Specifically, they proposed two kinds of abstract compilers to achieve this. 
The first one is to generate specialized analysis as a textual program, then 
can be loaded and executed by \texttt{eval} or similar mechanisms. The second 
one is using closure as a representation of specialized analysis. 
A closure is a function that remembers the environment where it defined. So the 
abstract compiler generates higer-order functions on-the-fly rather than textual 
program, then the functions can be applied immediately in a higher-order host language.

\citeauthor{Johnson:2013:OAA:2500365.2500604} in their work also adopt the 
idea closure generation for optimizing small-step abstracting abstract machine
\cite{Johnson:2013:OAA:2500365.2500604}. The source program is firstly compiled
to intermediate form called "abstract bytecode", which are actually higher-order functions, 
and then executed on a abstract abstract machine.

\citeauthor{damian1999partial}'s work on partial evaluation for program analysis \cite{damian1999partial}.

\citeauthor{amtoft1999partial}'s work on partial evaluation for constraint-based 
control flow analysis \cite{amtoft1999partial}.

\textbf{Abstracting Abstract Machine}. Abstracting Definitional Interpreter \cite{darais2017abstracting}

Collapsing Towers of Interpreters\cite{Amin:2017:CTI:3177123.3158140}

\section{Conclusion}

%% Acknowledgments
\begin{acks}                            %% acks environment is optional
                                        %% contents suppressed with 'anonymous'
  %% Commands \grantsponsor{<sponsorID>}{<name>}{<url>} and
  %% \grantnum[<url>]{<sponsorID>}{<number>} should be used to
  %% acknowledge financial support and will be used by metadata
  %% extraction tools.
  This material is based upon work supported by the
  \grantsponsor{GS100000001}{National Science
    Foundation}{http://dx.doi.org/10.13039/100000001} under Grant
  No.~\grantnum{GS100000001}{nnnnnnn} and Grant
  No.~\grantnum{GS100000001}{mmmmmmm}.  Any opinions, findings, and
  conclusions or recommendations expressed in this material are those
  of the author and do not necessarily reflect the views of the
  National Science Foundation.
\end{acks}

%% Bibliography
\bibliography{references}

%% Appendix
\appendix
\section{Appendix}

Text of appendix \ldots

\end{document}
