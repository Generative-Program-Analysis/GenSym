\section{From Interpreters to Abstract Interpreters} \label{unstaged_abs}

After seeing both unstaged and staged concrete interpreters, now we turn our
focus on abstract interpreters under the same framework. We will first present a
lattice framework with binding-time abstraction, but for the purpose of clearly
illustrating the idea, we will just use simple abstract domains such as power set
lattice or point-wise map lattice. Then we show the abstract components, namely
abstract values, abstract environments and abstract stores.
The abstract interpreter we present in this section is similar to
\citet{DBLP:journals/pacmpl/DaraisLNH17}'s, it is also
context/path/flow-insensitive by our choice.  In Section \ref{cfa}, we will
instantiate a commonly-used context-sensitive analysis;
\citet{Darais:2015:GTM:2814270.2814308} shows how to achieve path- and
flow-sensitivity by varying the monads, which is also applicable to our unstaged
or staged abstract interpreter.

\subsection{Stage Polymorphic Lattices} \label{stagedpoly_lat}

A complete lattice over set $S$ is a tuple of $\langle S, \sqsubseteq, \top,
\bot, \sqcup, \sqcap \rangle$, where $\sqsubseteq : S \to S \to Bool$ is the
ordering relation, $\sqcup: S \to S \to S$ is the join (least upper bound)
operator, and $\sqcap: S \to S \to S$ is the meet (greatest lower bound)
operator. The trait @SPLattice@ (Figure \ref{fig:splattice}) defines the type
class of stage polymorphic lattices. Similar to @MonadOps@ in Section
\ref{generic_if}, we introduce an additional higher-kinded type @R[_]@ to
annotate the binding-times, where @R[_]@ wraps @S@ and @Boolean@ in these
operations.

\begin{figure}[h!]
  \centering
  \begin{subfigure}[b]{0.45\textwidth}
  \begin{lstlisting}[style=small]
trait SPLattice[S, R[_]] {
  val ⊤: R[S];  val ⊥: R[S]
  def ⊑(l1: R[S], l2: R[S]): R[Boolean]
  def ⊔(l1: R[S], l2: R[S]): R[S]
  def ⊓(l1: R[S], l2: R[S]): R[S]
}
trait Lattice[S] extends SPLattice[S, NoRep]
  \end{lstlisting}
  \caption{trait \texttt{SPLattice} and \texttt{Lattice}} \label{fig:splattice}
  \end{subfigure}
  ~
  \begin{subfigure}[b]{0.6\textwidth}
\begin{lstlisting}[style=small]
def SetLattice[T]: Lattice[Set[T]] = new Lattice[Set[T]] {
  lazy val ⊥: Set[T] = Set[T]()
  lazy val ⊤: Set[T] = error("No representation for ⊤")
  def ⊑(l1: Set[T], l2: Set[T]): Boolean = l1 subsetOf  l2
  def ⊔(l1: Set[T], l2: Set[T]): Set[T]  = l1 union     l2
  def ⊓(l1: Set[T], l2: Set[T]): Set[T]  = l1 intersect l2
}
\end{lstlisting}
  \caption{The power set instance for lattice} \label{fig:powerset}
\end{subfigure}
\end{figure}

In this section, we simply let @R[T] = T@ (i.e., @NoRep@) to instantiate the
unstaged lattice type class. An simple instance of lattice we used is the power
set abstract domain (shown in Figure \ref{fig:powerset}), where the ordering is subset relation, the join is union
operation, and the meet is intersect operation.
Other lattices used in the remainder of the paper, such as products and maps, can be
implemented similarly, or lifted from existing lattices element-wise or
point-wise.

\subsection{Abstract Semantics}

\paragraph{Abstract Components}

% \cite{DBLP:conf/icfp/HornM10, DBLP:journals/jfp/HornM12}

We follow the \textit{abstracting definitional interpreters} (ADIs) approach
\cite{DBLP:journals/pacmpl/DaraisLNH17} to refactor our concrete interpreter to
an abstract interpreter. We first widen the concrete values to @AbsValue@: numbers
are lifted to a singleton object @IntTop@ representing the set of all numbers;
closures remain the same. Then the type @Value@ is sets of @AbsValue@ (we abuse
the type names a little bit, as the generic interface defined the abstract type
@Value@).
The address space is constrained to be finite to ensure the computatbility of
analysis -- for the simplest monovariant analysis, we use variable names for
addresses. After settling the @Value@ and @Addr@ type, the environments and
stores are automatically lifted to abstract environments and abstract stores,
i.e., @Map[Ident, Addr]@ and @Map[Addr, Set[AbsValue]]@ respectively.
It is important that the abstract store now maps addresses to sets of abstract
values, meaning an address may point to an over-approximated set of values
occurred at runtime.

\begin{lstlisting}
  trait AbstractComponents extends Semantics {
    sealed trait AbsValue
    case object IntTop extends AbsValue
    case class CloV(lam: Lam, env: Env) extends AbsValue
    type Value = Set[AbsValue]; case class Addr(x: String)
    ...
  }
\end{lstlisting}

To prevent the abstract interpreter from falling into non-termination, the ADI
approach uses a two-cache mechanism. Here we first provide the type
definitions, and later we will describe the algorithm in detail. A @Cache@ is a
mapping from configurations @Config@ to sets of value-store pairs, where the
configuration is a triple of current expression being evaluated, environment and
store.

\begin{lstlisting}
  type Config = (Expr, Env, Store); type Cache = Map[Config, Set[(Value, Store)]]
\end{lstlisting}

As of now, the store is a mapping from addresses to sets of abstract values.
Therefore, when applying a function, for instance @f(a)@, we may retrieve a set
of closures by dereferencing the address of @f@. We need to nondeterministically
apply all of the target closures.
Another nondeterminism happens when evaluating conditionals, we have to evaluate
both branches if the condition can not be determined. In other words, we will
have multiple computing results.

\begin{lstlisting}
  type Result = (R[Set[(Value, Store)]], R[Cache])
\end{lstlisting}

This observation leads us to the type definition of @Result@: in concrete setting,
the @Result@ is a pair of @Value@ and @Store@, now we use a set of such pairs
@Set[(Value, Store)]@ to represent multiple results. The results is also linked
with the resulting cache used in this iteration.
Note that the abstract binding-time annotation is wrapped to the two objects in
the @Result@ pair, but not the whole pair. Because we statically know the result
is a tuple of two elements, but we don't know how many value-store pairs we
would have, neither the content of the cache.

\paragraph{Monads for Abstract Interpretation} Now we incorporate the
nondeterminism and caching into the monad stack. For concrete evaluation, we
already have a reader effect for environments and a state effect for stores. We
further add three more effects based on it: the nondeterminism effect is
represented by the @SetT[M[_], A]@ monad transformer, inside of which, we have
a @ReaderT@ for a @in@ cache, and a @StateT@ for an @out@ cache. \todo{highlight
where changed}

\begin{lstlisting}
  trait AbstractSemantics extends AbstractComponents {
    type R[T] = T
    type AnsM[T] = ReaderT[StateT[SetT[ReaderT[StateT[IdM, Cache, ?], Cache, ?], ?], Store, ?], Env, T]
    ...
  }
\end{lstlisting}

The sketch implementation of @SetT@ transformer is shown below. The value @run@
it encapsulates is of type @M[SetT[A]]@, where @M[_]@ is the inner monad, and
@A@ the element type of the list. The @flatMap@ method first unwraps the inner
monadic value @M[SetT[A]]@ and obtain a value of @Set[A]@, then applies @f@ to
each element in that list, and finally folds the results back to a single
@SetT[M, B]@.

\begin{lstlisting}
  case class SetT[M[_]: Monad, A](run: M[Set[A]]) {
    def flatMap[B](f: A => SetT[M, B]): SetT[M, B] = ...
  }
\end{lstlisting}

\paragraph{Primitive Operations} Some of the primitive operations are changed
according to the new monad stack. A notable change is that the store update
operation joins the value with the existing values, for the same address. Under
abstract interpretation, we may have to explore the results from both branches
of a conditional expression for soundness. In @br0@, we combine the results
using @mplus@ from @MonadPlus@, which requires the values are join-semilattices
(in our case, @Set@s and @Map@s).

\begin{lstlisting}
  def set_store(αv: (Addr, Value)): AnsM[Unit] = liftM(StateTMonad.mod(σ => σ ⊔ Map(αv)))
  def br0(test: Value, thn: => Ans, els: => Ans): Ans = ReaderTMonadPlus.mplus(thn, els)
\end{lstlisting}

The value representation of lambda terms is still @CloV@, but we lift it to a
singleton set.

\begin{lstlisting}
  def close(ev: Expr => Ans)(λ: Lam, ρ: Env): Value = Set(CloV(λ, ρ))
\end{lstlisting}

For function applications, @ap_clo@ looks roughly the same as the concrete
interpreter. The difference is now the function @rator@ is a set of closures, so
we can not directly destruct it by pattern matching. Instead, we use an
auxiliary function @lift_nd@ that takes a set and lifts the elements into the
monad stack. Then we can still implement @ap_clo@ in monadic style, where the
@CloV@ comes from the @SetT@ and the nondeterminism is naturally handled.

\begin{lstlisting}
  def lift_nd[T](vs: Set[T]): AnsM[T]
  def ap_clo(ev: Expr => Ans)(rator: Value, rand: Value): Ans = for {
    CloV(Lam(x, e), ρ) <- lift_nd[AbsValue](rator)
    α <- alloc(x)
    _ <- set_store(α → rator)
    rt <- local_env(ev(e))(ρ + (x → α))
  } yield rt
\end{lstlisting}

%%%%%%%%%%%%%%%%%%%%%%%%%%%%%%%%%%%%%%%%%%%%%%%%%%%%%%%%%

\paragraph{Caching and Fixpoint Iteration}

As we mentioned earlier, the ADI approach uses a two-cache mechanism to
compute the least fixed-point and prevent non-termination, which is also called a
\textit{co-inductive} caching algorithm or \textit{truncated depth-first
evaluation} \cite{Rosendahl:AbsIntPL}. It has been widely used in abstract
interpretation or fixed point computation
\cite{DBLP:journals/pacmpl/DaraisLNH17, Wei:2018:RAA:3243631.3236800,
Rosendahl:AbsIntPL}. Since the @in@ and @out@ cache are wrapped in the monad
stack, we first defined several methods to obtain or update the @in@ and @out@
cache through the monad stack.

\begin{lstlisting}
  def ask_in_cache: AnsM[Cache]; def get_out_cache: AnsM[Cache]
  def put_out_cache(out: R[Cache]): AnsM[Unit]
  def set_out_cache(cfg: R[Config], vs: R[(Value, Store)]): AnsM[Unit]
\end{lstlisting}

Figure \ref{fig:coind_cache} shows the co-inductive caching algorithm that
closes the open-recursive @eval@.
The idea is to first set up two caches called @in@ and @out@ both of type
@Cache@. Then similar to Kleene's iteration algorithm, we compute the analysis
result iteratively. The @in@ cache contains what we already have computed from
the last iteration, and the @out@ cache is what we would have computed after this
iteration joined with the contents from @in@ cache.
When starting a new iteration, @in@ is set to be the result of the last
iteration, i.e., @out@; and @out@ is assigned to be empty.

During the iteration, we first check whether @out@ contains the configuration
@cfg@, representing the current desired computation, if
yes then result is directly returned through the monads.

Otherwise, we first retrieve the results from @in@ (@⊥@ if not contained in
@in@), and put it into the @out@ cache.
Then we invoke @ev@ to compute the result for this iteration.
After the evaluation, the store @σ@ may have been changed. So we update the
@out@ cache with the new result, i.e., @(v, σ)@. Similar to store update, cache
update is also point-wise join.
Note that for the recursive call of @ev@, we also instrument it by putting
@fix(ev)@ as the first argument to @ev@.

The iteration terminates when the resulting @out@ cache is equivalent to the
input @in@ cache, which means there is no more analysis result can be gained,
thus the iteration should end and we have reached the least fixed point.

\begin{figure}[t!]
  \centering
\begin{lstlisting}
  def fix(ev: (Expr => Ans) => (Expr => Ans)): Expr => Ans = e => for {
    ρ <- ask_env; σ <- get_store; in <- ask_in_cache; out <- get_out_cache; val cfg = (e, ρ, σ)
    rt <- if (out.contains(cfg)) for {
            (v, σ) <- lift_nd[(Value, Store)](out(cfg))
             _ <- put_store(σ)
          } yield v
          else for {
            _ <- put_out_cache(out + (cfg → in.getOrElse(cfg, ⊥)))
            v <- ev(fix(ev))(e)
            σ <- get_store
            _ <- update_out_cache(cfg, (v, σ))
          } yield v
  } yield rt
\end{lstlisting}
\vspace{-1em}
\caption{The unstaged co-inductive caching algorithm.}
\label{fig:coind_cache}
\end{figure}

\begin{lstlisting}
  def run(e: Expr): Result = fix(eval)(e)(ρ0)(σ0)(cache0)(cache0).run
\end{lstlisting}

Finally, we override the top-level @run@ method by running the monad with
the empty abstract environment $\rho_0$, initial store $\sigma_0$, and initial
caches @cache@$_0$.
