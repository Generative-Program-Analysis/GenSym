\section{From Interpreters to Abstract Interpreters} \label{unstaged_abs}

After seeing the unstaged and staged concrete interpreter, now we turn our
focus to abstract interpreters under the same framework. We first present a
lattice representation with binding-time abstraction, as well as simple
abstract domains, such as power sets.  Then, we show the abstract
components, namely, abstract values, abstract environments, and abstract stores.
The abstract interpreter we construct in this section is similar to
\citet{DBLP:journals/pacmpl/DaraisLNH17}'s. For simplicity, it is
context/path/flow-insensitive by our choice. In Section \ref{cfa}, we will
instantiate it to a commonly-used context-sensitive analysis.
\citet{Darais:2015:GTM:2814270.2814308} showed how to achieve path- and
flow-sensitivity by varying the monads, which is also applicable to our
unstaged or staged abstract interpreter.

\subsection{Stage Polymorphic Lattices} \label{stagedpoly_lat}

The abstract domains operated by abstract interpreters usually can be
formulated by complete lattices.
A complete lattice over set $S$ is a tuple $\langle S, \sqsubseteq, \top,
\bot, \sqcup, \sqcap \rangle$, where $\top$ is the top element, $\bot$ is the
bottom element, $\sqsubseteq : S \to S \to Bool$ is the
ordering relation, $\sqcup: S \to S \to S$ is the join (least upper bound)
operator, and $\sqcap: S \to S \to S$ is the meet (greatest lower bound)
operator. The trait @SPLattice@ (Figure \ref{fig:splattice}) defines a type
class of stage polymorphic lattices. Similar to the @MonadOps@ in Section
\ref{generic_if}, we introduce an additional higher-kinded type @R[_]@ to the
trait for annotating the binding-times, where @R[_]@ wraps the data type
@S@ and @Boolean@ in these operations.

\begin{figure}[h!]
  \centering
  \begin{subfigure}[b]{0.45\textwidth}
  \begin{lstlisting}[style=small]
  trait SPLattice[S, R[_]] {
    val ⊤: R[S];  val ⊥: R[S]
    def ⊑(l1: R[S], l2: R[S]): R[Boolean]
    def ⊔(l1: R[S], l2: R[S]): R[S]
    def ⊓(l1: R[S], l2: R[S]): R[S]
  }
  trait Lattice[S] extends SPLattice[S, NoRep]
  \end{lstlisting}
  \caption{trait \texttt{SPLattice} and \texttt{Lattice}} \label{fig:splattice}
  \end{subfigure}
  ~
  \begin{subfigure}[b]{0.6\textwidth}
\begin{lstlisting}[style=small]
  def SetLattice[T]: Lattice[Set[T]] = new Lattice[Set[T]] {
    lazy val ⊤: Set[T] = error("No representation for ⊤")
    lazy val ⊥: Set[T] = Set[T]()
    def ⊑(l1: Set[T], l2: Set[T]): Boolean = l1 subsetOf  l2
    def ⊔(l1: Set[T], l2: Set[T]): Set[T]  = l1 union     l2
    def ⊓(l1: Set[T], l2: Set[T]): Set[T]  = l1 intersect l2
  }
\end{lstlisting}
  \caption{The power set instance for lattice} \label{fig:powerset}
\end{subfigure}
\end{figure}

In this section, we simply instantiate @SPLattice@ with the flat binding-time
type @R[T] = T@ (i.e., @NoRep@); the result is the trait @Lattice[S]@.
An example of the lattices we used in the rest of the paper is the power set
abstract domain (shown in Figure \ref{fig:powerset}). For two power sets,
they are ordered by the subset relation. We use set union to compute their
join, and set intersection to compute their meet.  The bottom element of a
power set is empty set, and we do not have a representation of top element for
power set.  Other lattices used in the paper, such as products and maps, can be
implemented similarly or by lifting the existing lattices element-wise or
point-wise.  Non-relational numerical abstract domains such intervals can also
be implemented in a stage polymorphic way.

\subsection{Abstract Semantics}

In this section, we follow the abstracting definitional interpreters
(ADI) approach \cite{DBLP:journals/pacmpl/DaraisLNH17} to refactor our monadic
concrete interpreter to a monadic abstract interpreter.

\paragraph{Abstract Components}

We first widen the concrete value domain to @AbsValue@. There are two variants of
@AbsValue@: 1) numbers are lifted to a singleton object @IntTop@, which represents
the set of all numbers; 2) closures remain the same. Then, the type @Value@ is
redefined as a set of @AbsValue@ to account for approximation.  The address
space is constrained to be finite to ensure the computability of the analysis --
for a simple monovariant analysis, we directly use variable names for
addresses. After defining the @Value@ and @Addr@ type, the types of
environments and stores are automatically lifted to their abstract versions,
i.e., @Map[Ident, Addr]@ and @Map[Addr, Set[AbsValue]]@, respectively.  
\begin{lstlisting}
  trait AbstractComponents extends Semantics {
    sealed trait AbsValue
    case object IntTop extends AbsValue
    case class CloV(lam: Lam, env: Env) extends AbsValue
    type Value = Set[AbsValue]; case class Addr(x: String)
    ... }
\end{lstlisting}

It is important that the abstract stores now map addresses to sets of abstract
values, indicating that an address may point to an over-approximated set of
runtime values.  This can be justified by the approximation nature and
nondeterminism during the analysis.  For example, when analyzing a
conditional expression, we may not have enough information to decide which
branch will be taken, thus a sound treatment is to explore both of the branches.
Also, at some later point, we need to compute the join of two path.  Another
nondeterminism comes from function applications, for instance, @f(a)@, the
possible $\lambda$ terms of @f@ are computed approximately.  As the
result, we need to store all the possible callees, as well as we will
retrieve a set of closures when dereferencing an address. 

To ensure the abstract interpreter terminates on all programs when computing
the fixed-point, the ADI approach uses a co-inductive mechanism consisting of
two caches that remember the input and output of the @eval@ function.
Here, we first provide the necessary type definitions, and describe
the algorithm later. A @Cache@ is a mapping from configurations @Config@ to
sets of value-store pairs, where the configuration is a triple of the current
expression being evaluated, environment and store. Intuitively, a cache memoizes
the result values and stores for a given program configuration.
\begin{lstlisting}
  type Config = (Expr, Env, Store); type Cache = Map[Config, Set[(Value, Store)]]
\end{lstlisting}

\paragraph{Monads for Abstract Interpretation}
Compared with the concrete interpreter that uses reader and state effects, the
abstract interpreter further introduces the nondeterminism effect and another
reader and state effect for the two caches. The nondeterminism effect is
represented by the @Set[M[_], A]@ monad transformer, where @M@ is the inner
monad type being transformed, and @A@ is the type of elements in the set. 
We use a @ReaderT@ for one cache that is not changed during one fixed-point
iteration, and use a @StateT@ for another cache that will be constantly updated
during the analysis.
\citet{DBLP:journals/pacmpl/DaraisLNH17} discuss different permutations of the
monad stack for abstract interpretation. The following @AnsM@ type shows the
monad stack (we use \hl{light gray} to highlight what have been changed from
the monad stack of concrete interpretation):
\begin{lstlisting}[escapechar=!]
trait AbstractSemantics extends AbstractComponents {
  type R[T] = T
  type AnsM[T] = ReaderT[StateT[!\hl{SetT}![!\hl{ReaderT}![!\hl{StateT}![IdM, !\hl{Cache}!, ?], !\hl{Cache}!, ?], ?], Store, ?], Env, T]
  ... }
\end{lstlisting}

Similar to the concrete scenario, we sketch the @flatMap@ implementation of the
@SetT@ transformer and omit the rest. The field @run@ encapsulated by the monad
is of type @M[SetT[A]]@, where @M[_]@ is the inner monad, and @A@ is the element
type of the set. We first apply @flatMap@ on the inner monad to obtain the
set @s@. Then, we use the function @f@ to transform every element of type @A@
into a monadic value of type @Set[M, B]@. Finally, we fold all the transformed
values into a single monadic value of type @Set[M, B]@ by appending all of them.
The initial value of the fold is an empty @SetT@.
\begin{lstlisting}
  case class SetT[M[_]: Monad, A](run: M[Set[A]]) {
    def flatMap[B](f: A => SetT[M, B]): SetT[M, B] =
      SetT(Monad[M].flatMap(run) { (s: Set[A]) =>
        s.foldLeft(SetT.empty[M, B])((acc, a) => acc ++ f(a)).run
      }); ... }
\end{lstlisting}

This monad stack scheme (@AnsM[T]@) leads to a different type of grounded
result. Recall that in the concrete setting, we only have a reader monad (for
environments) and a state monad (for stores). The reader effect is not
persistent to the final result. Hence, together with the value produced by the
interpreter, the final result type is just a pair of @Value@ and @Store@.
However, under the new monad stack for abstract interpretation, we have a
@SetT@ monad transformer inside the environment monad and store monad. Therefore, the type
@Set@ becomes the container type of the pairs of values and stores, i.e.,
@Set[(Value, Store)]@.  Also, note that the reader and state monad for the
caches both inhibit internally of the nondeterminism monad; as the result, the
final result type pairs the set of value-stores with the cache type from
that state monad, as shown below.
\begin{lstlisting}
  type Result = (R[Set[(Value, Store)]], R[Cache])
\end{lstlisting}

\paragraph{Primitive Operations} The primitive operations are changed according
to the new monad stack scheme and value domains. One of the notable changes is
that the store update operator merges the new values with existing values.  As
we mentioned before, sometimes we have to explore the both branches for
conditionals: in method @br0@, we combine the results using the @mplus@
operation from @MonadPlus@, which requires the value domains are
join-semilattices (in our case, @Set@ and @Map@).
\begin{lstlisting}
  def set_store(αv: (Addr, Value)): AnsM[Unit] = liftM(StateTMonad.mod(σ => σ ⊔ Map(αv)))
  def br0(test: Value, thn: => Ans, els: => Ans): Ans = ReaderTMonadPlus.mplus(thn, els)
\end{lstlisting}

The value representation of lambda terms is still a defunctionalized closure of
type @CloV@, but we lift it to a singleton set of @AbsValue@ (remember that
type @Value@ is an alias of @Set[AbsValue]@).
\begin{lstlisting}
  def close(ev: Expr => Ans)(λ: Lam, ρ: Env): Value = Set(CloV(λ, ρ))
\end{lstlisting}

For function applications, @ap_clo@ looks the same as the concrete interpreter.
The difference is that now the first argument of @ap_clo@ is a set of closures
@funs@, so we can not directly destruct it to a syntactic @Lam@ by pattern
matching. Instead, we use an auxiliary function @lift_nd@ that takes a set and
lifts the elements in the set into the monad stack. Then we can
straightforwardly implement the @ap_clo@ in monadic style, where the closures
come from the monads and thus the nondeterminism can be naturally handled. The
light gray line shows what is added from the concrete @ap_clo@.
\begin{lstlisting}[escapechar=!]
  def lift_nd[T](vs: Set[T]): AnsM[T]
  def ap_clo(ev: Expr => Ans)(funs: Value, rand: Value): Ans = for {
    !\hl{CloV(Lam(x, e), $\rho$) <- lift\_nd[AbsValue](funs)}!
    α <- alloc(x)
    _ <- set_store(α → funs)
    rt <- local_env(ev(e))(ρ + (x → α))
  } yield rt
\end{lstlisting}

\paragraph{Caching and Fixpoint Iteration}
As we mentioned earlier, the ADI approach uses a two-cache mechanism to
compute the least fixed-point and prevent non-termination.
The caching algorithm is also called a \textit{co-inductive} caching or
\textit{truncated depth-first evaluation} \cite{Rosendahl:AbsIntPL}. It has
been used in other abstract interpreters or fixed-point computation
\cite{DBLP:journals/pacmpl/DaraisLNH17, Wei:2018:RAA:3243631.3236800,
  Rosendahl:AbsIntPL}. The idea is to use an @in@ cache and an @out@ cache during
the depth-first evaluation. The @in@ cache stores the result from the last
iteration, and @out@ cache is used for updating in the current iteration. In
the next iteration, the last @out@ cache will be served as the @in@ cache, and
an empty cache is plugged-in to the @out@ slot. Once the @out@ cache does not
contain any new information compared with the @in@ cache, the fixed-point is
reached.
In our monad stack, the @in@ and @out@ caches are modeled by the reader monad and
state monad, respectively. We first define several methods to manipulate
the two caches through the monad stack (the implementations are elided).
\begin{lstlisting}
  def ask_in_cache: AnsM[Cache]; def get_out_cache: AnsM[Cache]
  def put_out_cache(out: R[Cache]): AnsM[Unit]
  def set_out_cache(cfg: R[Config], vs: R[(Value, Store)]): AnsM[Unit]
\end{lstlisting}

The co-inductive caching algorithm is implemented as an instrumentation over the
@eval@ function (Figure \ref{fig:coind_cache}), and it also closes the open recursion.
The instrumentation works as follows. Initially, the two caches are both empty.
During the iteration, we first check whether the @out@ cache contains the
configuration @cfg@, which represents the current desired computation. If @out@
does contain @cfg@, then the result is directly returned throughout the monad
stack.
Otherwise, we first retrieve the result from @in@ (@⊥@ if @in@ does not contain
@cfg@), and update this result from @in@ into the @out@ cache in the fashion of
join. Then, we invoke @ev@ to evaluate the result for this iteration, where
@ev@ takes @fix(ev)@ as the self reference.  After the evaluation, the store @σ@
may have been changed, so we obtain the latest store and construct the
result @(v, σ)@, which will be used to update the @out@ cache.

The iteration terminates when the resulting @out@ cache is equivalent to the
input @in@ cache, which indicates that there is no more fact has been discovered.
Therefore, the iteration should end, and we have reached the least fixed-point.
\begin{lstlisting}
  def run(e: Expr): Result = fix(eval)(e)(ρ0)(σ0)(cache0)(cache0).run
\end{lstlisting}

Finally, we override the top-level @run@ method by running the monadic value
@fix(eval)(e)@ with the initial environment $\rho_0$, initial abstract
store $\sigma_0$, and initial caches @cache0@; all of which are empty.
\begin{figure}[t]
  \centering
\begin{lstlisting}
  def fix(ev: (Expr => Ans) => (Expr => Ans)): Expr => Ans = e => for {
    ρ <- ask_env; σ <- get_store; in <- ask_in_cache; out <- get_out_cache; val cfg = (e, ρ, σ)
    rt <- if (out.contains(cfg)) for { // ask if out already contains the desired result
            (v, σ) <- lift_nd[(Value, Store)](out(cfg));  _ <- put_store(σ)
          } yield v
          else for {
            _ <- put_out_cache(out + (cfg → in.getOrElse(cfg, ⊥)))
            v <- ev(fix(ev))(e);  σ <- get_store;  _ <- update_out_cache(cfg, (v, σ))
          } yield v
  } yield rt
\end{lstlisting}
\vspace{-1em}
\caption{The unstaged co-inductive caching algorithm.}
\label{fig:coind_cache}
\vspace{-1em}
\end{figure}

