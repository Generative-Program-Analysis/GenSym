\subsection{Abstract Syntax} \label{bg_lang}

We consider a tiny higher-order functional language \TLang, which is based on direct-style $\lambda$-calculus
with numbers, arithmetics, recursions and conditionals.
In Section~\ref{cases_imp}, we will add more imperative features to the language.
Since we are mostly interested in analyzing the dynamic behavior of the program, we disguised any static semantics and
type system. We also assume that input programs are well-typed and variables are distinct.
The abstract syntax is as follows:

\begin{lstlisting}
abstract class Expr
case class Lit(i: Int)
case class Var(x: String)
case class Lam(x: String, e: Expr)
case class App(e1: Expr, e2: Expr)
case class AOp(op: String, e1: Expr, e2: Expr)
case class Rec(x: String, e: Expr, body: Expr)
case class If0(cnd: Expr, thn: Expr, els: Expr)
\end{lstlisting}

We give the concrete semantics of \TLang by showing a big-step definitional interpreter.
The interpreter is a recursive function that takes the program AST, environment and store, and returns 
the evaluated value and the accompanied store. The environment is a mapping from identifiers to addresses,
and the store is a mapping from addresses to values. 
We use the store to model recursion and mutation in concrete semantics; it is also useful for polyvariant analysis.
This environment-and-store-passing style big-step interpreter is standard and can also be obtained by 
refunctionalizing \cite{DBLP:conf/ppdp/AgerBDM03, Wei:2018:RAA:3243631.3236800} a small-step CESK 
machine \cite{DBLP:conf/popl/FelleisenF87}.

\subsection{Generic Interface} 
Before showing the actual implementation of the concrete semantics, let us first consider
a generic interface for the interpreters, which is semantics-agnostic and can be 
"interpreted" by either concrete or abstract, unstaged or staged semantics of our choice.
The trait \texttt{Semantics} first declares several abstract type members, such as \texttt{Addr},
\texttt{Value}, \texttt{Env}, \texttt{Store} and returned type \texttt{Ans}. 
It also declares initial values for environments and stores, and few abstract methods that 
manipulate environments and stores such as @put@ and @get@.

\begin{lstlisting}
trait Semantics {
  type R[+_]
  type Ident = String;  type Addr;  type Value;  type Env
  type Store;  type Ans = (R[Value], R[Store])
  def get($\rho$: R[Env], x: Ident): R[Addr]
  def put($\rho$: R[Env], x: Ident, a: R[Addr]): R[Env]
  def get($\sigma$: R[Store], a: R[Addr]): R[Value]
  def put($\sigma$: R[Store], a: R[Addr], v: R[Value]): R[Store]
  def alloc($\sigma$: R[Store], x: Ident): R[Addr]
  val $\rho$0: R[Env]; val $\sigma$0: R[Store]
  def close(ev: EvalFun)($\lambda$: Lam, $\rho$: R[Env]): R[Value]
  def num(i: Lit): R[Value]
  def apply_closure(ev: EvalFun)
    (f: R[Value], arg: R[Value], $\sigma$: R[Store]): Ans
  def branch0(cnd: R[Value], thn: => Ans, els: => Ans): Ans
  def prim_eval(op: Symbol, v1: R[Value], v2: R[Value]): R[Value]
}  // to be continued
\end{lstlisting}

The actual semantics and operations of the interpreter are left to be implemented later, 
these cinclude operations such as lifting literal terms to values (\texttt{close}, \texttt{num}), 
applying a closure (\texttt{apply\_closure}), branching (\texttt{branch0}),
and arithmetics (\texttt{prim\_eval}). We can choose to implement them in a way that is concrete or abstract,
unstaged or staged.
But, just based on these abstract operations, an expressive generic interpreter can be defined:

\begin{lstlisting}
trait Semantics {  // continued
  type EvalFun = (Expr, R[Env], R[Store]) => Ans
  def eval(ev: EvalFun)(e: Expr, $\rho$: R[Env], $\sigma$: R[Store]): Ans = 
    e match {
      case Lit(i) => (num(Lit(i)), $\sigma$)
      case Var(x) => (get($\sigma$, get($\rho$, x)), $\sigma$)
      case Lam(x, e) => (close(ev)(Lam(x, e), $\rho$), $\sigma$)
      case App(e1, e2) =>
        val (e1v, e1$\sigma$) = ev(e1, $\rho$, $\sigma$)
        val (e2v, e2$\sigma$) = ev(e2, $\rho$, e1$\sigma$)
        apply_closure(ev)(e1v, e2v, e2$\sigma$)
      case Rec(x, f, body) =>
        val $\alpha$ = alloc($\sigma$, x)
        val $\rho$_* = put($\rho$, x, $\alpha$)
        val (fv, f$\sigma$) = ev(f, $\rho$_*, $\sigma$)
        val $\sigma$_* = put(f$\sigma$, $\alpha$, fv)
        ev(body, $\rho$_*, $\sigma$_*)
      case If0(cnd, thn, els) =>
        val (cndv, cnd$\sigma$) = ev(cnd, $\rho$, $\sigma$)
        branch0(cndv, ev(thn, $\rho$, cnd$\sigma$), ev(els, $\rho$, cnd$\sigma$))
      case AOp(op, e1, e2) =>
        val (e1v, e1$\sigma$) = ev(e1, $\rho$, $\sigma$)
        val (e2v, e2$\sigma$) = ev(e2, $\rho$, e1$\sigma$)
        (prim_eval(op, e1v, e2v), e2$\sigma$)
    }
  def fix(ev: EvalFun => EvalFun): EvalFun = 
    (e, $\rho$, $\sigma$) => ev(fix(ev))(e, $\rho$, $\sigma$)
  def eval_top(e: Expr): Ans = eval_top(e, $\rho$0, $\sigma$0)
  def eval_top(e: Expr, $\rho$: R[Env], $\sigma$: R[Store]): Ans = 
    fix(eval)(e, $\rho$, $\sigma$)
}
\end{lstlisting}

The implementation of the skeleton interpreter traverses the abstract syntax tree and should be straightforward to understand.
It is worth noting that the interpreter is written in open-recursive style -- 
it can not refer to itself directly, instead, \texttt{eval} takes an additional 
parameter \texttt{ev} referring to itself; this allows us to instrument the interpreter
from the outside. Accordingly, the function \texttt{close} for lifting lambda term to closure values
is also written in this style.
We define a default implementation \texttt{fix} to close the recursion by recursively using the result 
of \texttt{fix(ev)} for \texttt{ev}. 
Finally, the top-level evaluation function \texttt{eval\_top} is defined.

\paragraph{Stage Polymorphism.}
Readers may notice that an abstract higher-kinded type \texttt{R[+T]} is defined in the trait, i.e. a
type-level function whici will be used to specify the binding-time, i.e., staged or not; the annotation 
\texttt{+} indicates that the type parameter \texttt{T} is covariant.
The LMS framework we use here allows us to specify binding-times purely based on types.
All occurrences of type \texttt{Env}, \texttt{Store}, etc. are wrapped with \texttt{R} 
essentially because they may potentially be dynamic when staging is introduced.
Obviously, the unstaged higher-kinded type then can be simply instantiated as an identity type function: 
\texttt{type R[+T] = T}. In Section~\ref{stagedinterp}, \texttt{R} will be instantiated 
as a staging annotation. 
%\todo{Finally tagless, poly embedding}

\subsection{Concrete Instantiation} \label{unstaged_conc}
Now we can instantiate a concrete interpreter. We first concretize the abstract types as follows.
Since the interpreter is concrete, for the propose that properly models the heap, the type \texttt{Int} 
is used for address space \texttt{Addr}. The environment and store are just ordinary \texttt{Map}s in Scala.
A value can be either a tagged number \texttt{NumV} or a closure \texttt{CloV} that contains a lambda term
and an environment.

\begin{lstlisting}
trait Concrete extends Semantics {
  type Addr = Int;   sealed trait Value
  case class CloV($\lambda$: Lam, $\rho$: Env) extends Value
  case class NumV(i: Int) extends Value
  type Env = Map[Ident, Addr];  type Store = Map[Addr, Value]
}
\end{lstlisting}

we proceed to implement the components that define the semantics. 
As mentioned before, type \texttt{R} is concretized as an identity type function, thus
all types are just ordinary types.

\begin{lstlisting}
object ConcInterp extends Concrete {
  type R[+T] = T
  val $\rho$0 = Map[Ident,Addr]();  val $\sigma$0 = Map[Addr,Value]()
  def get($\rho$: Env, x: Ident): Addr = $\rho$(x)
  def put($\rho$: Env, x: Ident, a: Addr): Env = $\rho$ + (x -> a)
  def get($\sigma$: Store, a: Addr): Value = $\sigma$(a)
  def put($\sigma$: Store, a: Addr, v: Value): Store = $\sigma$ + (a -> v)
  def alloc($\sigma$: Store, x: Ident): Addr = $\sigma$.size + 1
  def close(ev: EvalFun)($\lambda$: Lam, $\rho$: Env): Value = CloV($\lambda$, $\rho$)
  def num(i: Lit): Value = NumV(i.i)
  def apply_closure(ev: EvalFun)
    (f: Value, arg: Value, $\sigma$: Store): Ans = f match {
      case CloV(Lam(x, e), $\rho$) =>
        val $\alpha$ = alloc($\sigma$, x); val $\rho$_* = put($\rho$, x, $\alpha$)
        val $\sigma$_* = put($\sigma$, $\alpha$, arg); ev(e, $\rho$_*, $\sigma$_*)
    }
  def branch0(cnd: Value, thn: => Ans, els: => Ans): Ans = 
    cnd match { case NumV(i) => if (i == 0) thn else els }
  def prim_eval(op: Symbol, v1: Value, v2: Value): Value = 
    op match {
      case '+ => v1.asInstanceOf[NumV].i + v2.asInstanceOf[NumV].i
      ...
    }
}
\end{lstlisting}

Finally, \texttt{get} and \texttt{put} directly manipulate the map of environment and store. 
An allocation function is used when we need to update the store: 
in the concrete case, \texttt{alloc} always returns a fresh address of the store.
\texttt{close} and \texttt{num} lift syntactic literals to values.
\texttt{apply\_closure} takes the function value, argument value and the latest store,
then extends the environment and store after allocating a fresh address, and continue
evaluating the body \texttt{e} of the lambda term.
